\documentclass[]{imc-inf}
\usepackage[]{indentfirst}
\usepackage{enumitem} % Required for custom labelingHey, Cortana. 
\usepackage{amsmath}
\usepackage{array} % For table formatting with \arraybackslash
\usepackage[round,authoryear]{natbib} % For APA-style citations with parentheses
\setcitestyle{authoryear,round,citesep={;},aysep={,},yysep={,}} % APA 7 formatting
\bibliographystyle{apalike} % APA-like style with (Author, Year) format
\usepackage{booktabs}
\usepackage{longtable}
\usepackage{float}

\def\tightlist{}

\begin{document}
	
	\title{SAIL -- A Strategic AI Lifecycle Framework}
	\subtitle{Coordinated AI Adoption in Software Organizations}
	\thesistype{Master Thesis}
	\author{David Bobek}
	\supervisor{[Roger Hage]}
	\copyrightyear{2026}
	\submissiondate{March 2026}
	\keywords{AI Adoption, Enterprise Architecture, Digital Transformation, Innovation Diffusion Theory, Dynamic Capabilities, MLOps, AI Governance}
	
    \frontmatter
    \maketitle{}
    \addtoToC{Table of Contents}%
	\tableofcontents%
	\clearpage

	\begin{declarations} \end{declarations}
    	
    
    \begin{acknowledgements}
        [Your acknowledgements here]
    \end{acknowledgements}

	\begin{abstract}
        The rapid adoption of Artificial Intelligence (AI) technologies is transforming the way software organizations operate. While AI offers significant potential for innovation, efficiency, and competitive advantage, many organizations struggle to integrate AI solutions effectively into their existing workflows. Challenges include aligning AI initiatives with strategic objectives, managing the AI lifecycle, ensuring governance, and fostering organizational readiness.
		
		This thesis introduces \textbf{SAIL}, the \textit{Strategic AI Lifecycle} framework, designed to guide software organizations in the coordinated adoption of AI. SAIL provides a structured approach to managing AI initiatives, from strategic planning to deployment and continuous monitoring, ensuring that AI adoption is both effective and sustainable. By leveraging this framework, organizations can systematically address technical, operational, and strategic challenges associated with AI, enabling better decision-making and innovation.
		
		The research develops a comprehensive framework for AI adoption in software organizations, provides practical guidelines for managing the AI lifecycle, and highlights best practices and lessons learned for coordinated AI implementation.
	\end{abstract}

	\mainmatter

\chapter{Introduction}
	
	The rapid adoption of Artificial Intelligence (AI) technologies is transforming the way software organizations operate. While AI offers significant potential for innovation, efficiency, and competitive advantage \citep{McKinsey2024b, Dzreke2024}, many organizations struggle to integrate AI solutions effectively into their existing workflows \citep{Deloitte2025, Haefner2023}. Challenges include aligning AI initiatives with strategic objectives, managing the AI lifecycle, ensuring governance, and fostering organizational readiness \citep{IBM2024, Ransbotham2020}. Organizations face significant barriers in scaling AI beyond initial pilots, with issues ranging from technical complexity to organizational resistance \citep{Makarius2020}.
	
	This thesis introduces \textbf{SAIL}, the \textit{Strategic AI Lifecycle} framework, designed to guide software organizations in the coordinated adoption of AI. SAIL provides a structured approach to managing AI initiatives, from strategic planning to deployment and continuous monitoring \citep{Stone2025}, ensuring that AI adoption is both effective and sustainable. By leveraging this framework, organizations can systematically address technical, operational, and strategic challenges associated with AI, enabling better decision-making and innovation.
	
	\section{Research Objectives}
	
	The primary objectives of this thesis are to:
	\begin{itemize}
		\item Develop a comprehensive framework for AI adoption in software organizations.
		\item Provide practical guidelines for managing the AI lifecycle.
		\item Evaluate the effectiveness of the SAIL framework through case studies or simulations.
		\item Highlight best practices and lessons learned for coordinated AI implementation.
	\end{itemize}

	Despite the growing body of research done on air maturity models, digital transformation and enterprise architecture, the. Already existing approaches in the current state of the industry largely fail to provide an orchestrated, holistic approach on AI adoption and full lifecycle management. The current. Frameworks are not sufficient to guide software organizations from early AI experimentation to scalable organization wide. AI adoption. Most of the already existing models focus simply on analysis and assessment rather than a full execution. They often lack mechanisms for reflecting on processes and fail to encompass iterative approach.
	
	\section{Research Questions}
	Will be guided by the following research questions:

	\begin{enumerate}[label=\textbf{RQ\arabic*:}, leftmargin=1.5cm]
    \item \label{rq:lifecycle} How can AI adoption in software organizations be structured across the full lifecycle from opportunity identification to scalable, AI-native integration?
    
    \item \label{rq:mechanisms} What organizational, architectural, and governance mechanisms are required to enable coordinated and scalable AI adoption?
    
    \item \label{rq:framework} How can a lifecycle-oriented framework support decision-making, reuse, and alignment across multiple AI initiatives?
\end{enumerate}


\chapter{Literature Review}\label{literature-review}

In order to ground the SAIL framework in established theory, this
chapter will focus on the current state of research across five key
theoretical lenses relevant to AI adoption in software organizations.
The concept of AI adoption is multifaceted, involving technological,
organizational, and strategic dimensions \citep{Dwivedi2021}. The main challange is the lack
of an integrated framework which is addressing the concept of ``Full
Lifecycle'' adoption of AI in a coordinated and scalable manner \citep{Nogare2025}. The
aspect of AI being a very new technology, hinders the existence of
established frameworks and supportive literature \citep{Benbya2021}. Organizations require comprehensive guidance that addresses both technical and managerial aspects of AI adoption \citep{Berente2021}. Therefore, this
chapter will rather review individual theories and models that are going
to be synthesized later throughou the theoretical framework chapter.

\section{Enterprise Architecture and AI Integration}

In order to ground the SAIL framework in established theory the focus on
Enterprise Architecture (EA) is crucial. Enterprise Architecture is
providing a holistic approach to the problem of aligning IT with
business strategy and goals. The principle of EA and its utilization in
organizations is well documented in literature. However the
not so well pointed out aspect is the type of company to which EA is
applied. Different types of companies have different needs and
requirements and technological readiness. Software organizations are
typically more agile and introduction of major changes is easier to
implement and does not create a lot of resistance \citep{Kaddoumi2022}. In contrast to this,
traditional industries such as manufacturing or banking are more rigid
and changes are harder to implement, thus EA frameworks in these
industries are more rigid and bureaucratic \citep{Canat2018}.

EA does not work as a silver bullet and it is not solving all problems
by itself. There are several frameworks and methodologies which are used
in practice. The most well known and implemented in companies are
The Open Group Architecture Framework (TOGAF) \citep{TheOpenGroup2022} and the Zachman
Framework \citep{Zachman1987}. The mentioned frameworks are the current state of the art in
EA and are widely used in practice. Each of the frameworks has its
strengths and weaknesses and the choice of the framework depends on the
context of the organization. TOGAF being more dominant in the cases
where the organization is more agile and dynamic, while Zachman
Framework is more suitable for traditional and rigid organizations.

In the current Digital Transformation era, TOGAF is more suitable as
more and more organizations are becoming agile and dynamic and the usage
of AI can be implemented in a way that is more aligned with TOGAF
principles while still keeping the structure and rigor of EA \citep{Fitriani2023}. The usage
and value extraction of AI in organizations is not a trivial task and it
requires a structured approach. EA can provide the necessary structure
and rigor to ensure that AI initiatives are aligned with business
strategy and goals \citep{Ern2025}.

EA is therefore crucial for AI integration and scaling :

\begin{itemize}
	\item
	It alligns AI initiatives with business strategy, ensuring that
	isolated pilots are connected to broader organizational goals and
	prevents existance of ``siloed'' AI projects \citep{McKinsey2024b}.
	\item
	It provides a systemic view of the organization while having
	established governance mechanisms, which is crucial for managing the
	complexity of AI adoption across multiple teams and departments.
	\item
	It facilitates reuse of AI assets (models, data, knowledge) across
	teams by establishing common standards and repositories, thus
	preventing duplication of efforts and promoting efficiency \citep{Lakarasu2022}.
\end{itemize}

On contrast the framework of Zachman is more rigid and bureaucratic and
it is not so suitable for the dynamic nature of AI adoption. The Zachman
points out that the architecture is a schema for organizing
architectural artifacts (for example, design documents, specifications,
and models) that takes into account both who the artifact targets (for
example, business owner and builder) and what particular issue (for
example, data and functionality) is being addressed \citep{RobertsonDunn2012}. The Zachman
framework is more suitable for traditional and rigid organizations where
changes are harder to implement. The usage of AI in such organizations
is more challenging as the adoption of AI requires a more agile and
dynamic approach.

Both of the frameworks have their strengths and weaknesses and the
choice of the framework depends on the context of the organization.
However, for the purpose of this thesis, TOGAF is more suitable as it is
more aligned with the principles of Digital Transformation and AI
adoption. Both of them were developed in the times of ``pre-AI'' era and
therefore do not address the specific challenges and requirements of AI
adoption. However, the principles and concepts of EA can be adapted and
extended to address the specific needs of AI adoption in software
organizations. The SAIL framework will therefore build upon the
principles of EA and adapt them to the specific context of AI adoption
in software organizations.

\section{Digital Transformation and Organizational Readiness}

The concept of Digital Transformation is closely related to the adoption
of AI in organizations. Digital Transformation refers to the integration
of digital technologies into all areas of a business, fundamentally
changing how the organization operates and delivers value to customers \citep{Westerman2019}.
Year by year, more and more organizations are undergoing Digital
Transformation in order to stay competitive in the market and the
prevalence of AI is only amplifying this successful digital revolution.
The proof of the fact that Digital Transformation is in fact a
successful concept can be seen by comparing the performance of companies
which have undergone Digital Transformation in the last decade. Digital leaders
achieve earnings growth that is 1.8 times higher than digital laggards, and more than
double the growth in total enterprise value \citep{BCG2020}. The companies which have undergone
Digital Transformation have seen a significant increase in revenue and
market share, while the companies which have not undergone Digital
Transformation have seen a decline in revenue and market share.

The success of Digital Transformation can be observed in the performance
of companies that have succesfully embraced it compared to those that
have not. Firms that have implemented Digital Transformation strategies
report significantly stronger revenue growth and improved market share \citep{DeloitteInsights2023}.
Research shows that digital leaders achieve substantially higher earnings growth
and enterprise value compared to their less digitally mature peers \citep{BCG2020}.
However, the path to success is challenging: 70\% of digital transformations 
fall short of their objectives, with only 30\% achieving their targets and
creating sustainable change \citep{BCG2020}.

These results show that Digital Transformation is not just a theory; it
is a real way to gain a competitive edge. As AI technologies become more
and more a part of Digital Transformation, they are expected to help
businesses make more money, run more smoothly, and stay strong over
time.

Proof of Digital Transformation success:

\begin{itemize}
	\item Digital leaders achieve earnings growth that is 
	1.8 times higher than digital laggards, and more than double the growth in 
	total enterprise value \citep{BCG2020}. Organizations must carefully select appropriate 
	metrics to measure digital transformation progress, as wrong KPIs can lead to 
	misaligned efforts and failed outcomes \citep{Ahmad2021, Schrage2022}.
	
	\item Successful digital transformations create, on average, 66\% more value 
	and improve corporate capabilities by 82\% compared to failed transformations
	\citep{BCG2020}.
	
	\item Only 30\% of digital transformations meet or exceed their target value 
	and result in sustainable change, while 70\% fall short of their objectives
	\citep{BCG2020}.
	
	\item Organizations that adopt AI as part of their digital transformation 
	strategy gain significant competitive advantages in innovation, efficiency, 
	and market positioning \citep{McKinsey2024a}.
\end{itemize}


Regarding AI adoption, Digital Transformation provides several lessons:
- Organizational readiness is crucial. Successful Digital Transformation
requires not just technology but also changes in culture, skills, and
processes \citep{Jöhnk2021}. Similarly, AI adoption needs organizational readiness to
manage change effectively \citep{Hradecky2022, Holmstrom2022}. Building AI-powered organizations requires systematic transformation of leadership, technology, data, and organizational capabilities \citep{Fountaine2019}. - Maturity models can guide progression.
Digital Transformation maturity models outline stages from initial
experimentation to full integration. AI adoption can benefit from
similar staged approaches to manage complexity and scale effectively \citep{Enholm2022}.

The implementation of AI in organizations very much depends on the
readiness of the organization to embrace change \citep{Shrestha2019}. This change comes in
different forms, such as changes in culture, skills, and processes but
also in the current technological infrastructure of the company and its
ability to integrate new technologies. Challanges such as lack of
permissions, lack of skills, lack of understanding of the technology and
its potential, lack of resources and budget are all common barriers to
the adoption of AI in organizations \citep{Bain2024}. These barriers can be overcome by
following a structured approach to AI adoption, which is the main
purpose of the SAIL framework. This would allow companies to overcome
the barriers and successfully implement AI in their organizations and
show them the standardized way of doing so and thus increase the chances
of success while outlining which fields of the organization need to be
changed in order to successfully implement AI.

The broad term of organizational readiness can be measured through
different maturity models \citep{AlAli2022}. The most well known and widely used maturity
model is the Digital Maturity Model (DMM) developed by Deloitte \citep{DeloitteInsights2023}. The DMM
outlines five stages of digital maturity: 1. Initial: Ad hoc and
uncoordinated digital initiatives. 2. Developing: Some digital
initiatives, but still siloed and uncoordinated. 3. Defined: Digital
initiatives are defined and coordinated across the organization. 4.
Managed: Digital initiatives are managed and measured for impact. 5.
Optimized: Digital is fully integrated into business strategy and
operations.

In order to successfully implement AI in organizations, the assumption
of high level of digital maturity is crucial. The higher the level of
digital maturity, the higher the chances of success in implementing AI
in organizations. The SAIL framework will therefore build upon the
principles of Digital Transformation and adapt them to the specific
context of AI adoption in software organizations.

\section{Innovation Diffusion Theory}

Innovation Diffusion Theory (IDT), developed by Everett Rogers, provides
a valuable lens for understanding how new technologies, such as AI, are
adopted within organizations \citep{Rogers2003}. This theory is particularly relevant in
the context of AI adoption in software organizations, where the pace of
rapid industry change and technological advancement creates both
opportunities and challenges for adoption.

IDT become relevant in the context of technology adoption in
organizations. The theory outlines how innovations are communicated over
time among the members of a social system. The key elements of IDT
include the innovation itself, communication channels, time, and the
social system \citep{Dearing2009}. The rise of usage of IDT is closely related to the rise
of technology adoption in organizations and the recognition that
successful adoption of new technologies requires more than just the
technology itself; it also requires effective communication and social
dynamics within the organization \citep{Almaiah2022}.

Perfect example where IDT was successfully applied is in the Israeli
tech sector, where early adopters in startups and tech companies drove
the initial adoption of AI technologies, leading to broader acceptance
across the industry. This diffusion was facilitated by strong networks
and knowledge sharing among innovators and early adopters, which helped
to build trust and demonstrate the value of AI. Thus creating a country
which swiftly became the ``Startup Nation'' and a global leader in AI
innovation while having an insignificant population of only 9 million
people and the amount of AI startups per capita being the country with
the highest number worldwide \citep{StartupGenome2024, StartupNationCentral2024}.


The key components of IDT include: - Innovation: The acceptance of AI
technologies as a new innovation within the organization. -
Communication Channels: The methods through which information about AI
is shared within the organization (e.g., meetings, workshops, internal
communications). - Time: The duration over which AI adoption occurs,
including the stages of adoption. - Social System: The organizational
culture and structure that influence how AI is perceived and adopted.

These key components were laid out in Rogers' seminal work ``Diffusion
of Innovations'' (2003) more than two decades ago, but they remain
highly relevant in today's context of rapid technological change and AI
adoption. The ability to understand and leverage these components is
crucial for successful AI adoption in software organizations \citep{Schneider2023}. Each of
these components plays a critical role in shaping the adoption process
and determining the success of AI initiatives. By following the
principles of IDT, organizations are able to effectively manage the
adoption process, address resistance, and build momentum for AI
initiatives, while still being aware of the risks and challenges
associated with AI adoption. The SAIL framework will therefore
incorporate insights from IDT to guide the scaling strategies for AI
adoption in software organizations. The effective usage of IDT in
organisations can be implemented through the following steps:

\begin{enumerate}
	\def\labelenumi{\arabic{enumi}.}
	\item
	Start with innovators and early adopters: Identify and engage
	individuals or teams within the organization who show a strong
	interest in AI and are willing to experiment with new technologies.
	This will naturally create champions for AI adoption. The most
	important aspect is to identify the right people who are willing to
	experiment and take risks and give the people the freedom to do so.
	This will create a culture of innovation and experimentation within
	the organization.
	\item
	Leverage communication channels: The usage of effective communication
	channels is crucial for spreading awareness. The spark that was
	created by the innovators and early adopters needs to be spread across
	the organization and let the fire of innovation spread. The most
	effective ways to do so are through company wide presentantions,
	demos, workshops and upper management support. The communication needs
	to be clear and concise and the benefits of AI adoption need to be
	highlighted.
	\item
	Address concerns and resistance: It is natural for people to be
	resistant to change, the natural human behavior is to resist change
	and stick to the status quo. Therefore the emphasize on addressing
	concerns and resistance is crucial. This is done by creating a level
	of trust and transparency. The concerns of employees need to be
	addressed and the benefits of AI adoption need to be highlighted. For
	the successful adoption of AI the value proposition needs to be clear
	and the benefits need to outweigh the risks \citep{Patnaik2024}.
	\item
	Build a supportive social system: Organizational culture and structure
	play a crucial role in shaping how AI is perceived and adopted. This
	means that the culture of the organization needs to be supportive of
	innovation and experimentation. By fostering a culture that is
	exploratory and innovation driven we can create an environment where
	AI adoption can thrive. The kernel of the culture needs to allign with
	the long term mission and vision of the organization which at the end
	needs to be AI-Native.
\end{enumerate}

By following the principles of IDT, organizations have a clear way to
manage the adoption process, address resistance, and create a momentum
where AI is embraced and incorporated into the culture which the SAIL
framework will build upon.

\section{Dynamic Capabilities Theory}

Dynamic Capabilities Theory (DCT), introduced by David Teece \citep{Teece1997}, provides a
valuable framework for understanding how organizations can adapt and
thrive in rapidly changing environments such as the current AI landscape
which is mostly characterized by extremely fast technological
advancements and changes. DCT as a theory is particularly emphasizing
the importance of an organization's ability to sense, seize, and
transform in order to maintain a competitive advantage. The winner is
not the strongest or the most intelligent, but the one who is most
adaptable to change while still achieving a sustained competitive
advantage. DCT was origanlly developed in the context of strategic
management and organizational theory, but its wide adoption in the tech
and software industry is closely related to the rapid pace of
technological change and the need for organizations to become more
agile, adaptive and innovative in order to stay competitive.

DCT consists of three main components: 1. Sensing: The ability to
identify and assess opportunities and threats in the external
environment. 2. Seizing: The ability to mobilize disposable resources to
capture value from opportunities. 3. Transforming: The ability to
continuously renew and realign organizational processes and structures \citep{Liu2024}
to adapt to changing environments while still maintaining operational
efficiency.

The winners in the modern digital economy are those who can quickly
sense new opportunities, seize them effectively, and transform their
organizations to stay ahead of the competition. These 3 simple steps
might seem trivial, but they are the key to success. This type of mental
allignment will be extremely relevant for the successful adoption of AI
in organizations. By not following these principles, organizations risks
moving and operating outside of their core competencies and thus
deviating from their core mission and vision. Having a clear 3 step
approach helps the organization to stay focused and aligned while still
being able to reflect and adapt to the changing environment.

DCT can be used in both small and large organizations, but the
implementation might differ. In small organizations, the decision-making
process is typically more centralized, allowing for quicker sensing and
seizing of opportunities. On the other hand the lack of available and
disposable resources might hinder the ability to seize opportunities
effectively and thus leading to a more visionary mindset instead of an
operational one. In large organizations, the decision-making process is
typically more decentralized and complex, which in most cases leads to
slower sensing and seizing of opportunities. The advantage of large
organizations is the availability of resources which allows them to
seize opportunities more effectively and while not being as visionary as
small organizations, they are more operationally efficient and the
ability to transform and capitalize on opportunities is higher.

The usage of DCT can be perfectly showcased on an rather succesful
startup from the heart of Tel Aviv called ``Waze''. Waze was founded in
2006 and was acquired by Google in 2013 for \$1.1 billion. The success
of 3 Jewish founders Ehud Shabtai, Amir Shinar, and Uri Levine can be
attributed to their ability to sense the opportunity of real-time
traffic information, seize the opportunity by developing a mobile app
that leveraged user-generated data, and transform the organization by
continuously updating and improving the app based on very crucial user
feedback. Organizations developing AI capabilities must similarly develop dynamic capabilities that enable them to sense opportunities, seize them through AI implementation, and transform their operations to maintain competitive advantage \citep{Mikalef2021, Wamba2021}.

\section{Related AI Adoption Models}

The AI adoption models that are currently existing in literature and
practice are mostly focused on the maturity of AI adoption in
organizations. These models are typically structured around different
levels of maturity and their intended use is to help organizations
assess their current state of AI adoption and identify areas for
improvement. This is a very useful approach, which can help
organizations where to start and what are the prerequisites for
successful AI adoption. However, these models are mostly focused on the
maturity of AI adoption and do not provide a structured approach to the
actual adoption process itself. This is the research and market gap
where the SAIL framework is trying to fill in.

Several AI maturity models have been developed by leading consulting firms and 
research organizations to help organizations assess their AI readiness and capabilities.
The most prominent example is Deloitte's AI Maturity Model, which outlines five levels 
of AI maturity, from ``Ad Hoc'' to ``Optimized'', focusing on areas such as strategy, 
governance, data management, and talent \citep{DeloitteInsights2023}. Similar frameworks have been 
proposed by other major consulting firms, typically structured around progressive maturity 
stages that organizations can use for self-assessment. Academic research has contributed 
additional maturity models, including frameworks that progress from isolated ignorance to 
systematic intelligence \citep{Lichtenthaler2020} and organizational readiness assessment 
frameworks that examine the structural prerequisites for AI adoption \citep{Pumplun2019}. However, a common limitation across 
these models is that while they are useful for assessing the current state of AI adoption, 
they generally do not provide a detailed, structured approach to the adoption process itself.
Academic research on digital maturity in specific sectors, such as education, has also 
contributed frameworks that can be adapted to AI adoption contexts \citep{AlAli2022}.

These models are
certainly useful for organizations in order to approach the topic of AI
adoption and understand where they currently stand. Every company and
business operating in the market needs to start somewhere and these
models provide a good starting point. However, the main question is
still opened and that is ``What next?'' and ``How to actually implement
AI in a structured way?''. This question is still not answered by the
existing models and is the most crucial one as the actual conversion of
the theoretical knowledge into practical implementation is the most
challenging part, but is the only part which actually creates value.
Staying too long undercover with reasearch gets you only as far as the
size of the company's budget allows you to go. The actual implementation
and value creation is the only thing which will make the company
successful in the long run. The purpose of the SAIL framework is to
convert the theoretical knowledge into practical implementation and
provide a structured approach to the adoption process itself. While
focusing on mostly internal aspects of the organization and efficiency,
resulting in reduction of overhead costs and increase of productivity.

\section{Already Existing Solutions}\label{already-existing-solutions}

This section this section is going to review all of the existing AI adoption in terms of their scaling solution, how the research is implemented, their frameworks across various sectors. The aim is to provide the current state of the industry and documented approaches.


\subsection{Enterprise Infrastructure and Deployment Solutions}

The research conducted over the last few years has placed AI deployment and integration as a full life cycle problem \citep{Hechler2020, Sharma2024, Ali2018}. It encompasses the processes of design, DevOps, and governance pillars. It outlines that there are significant gaps when transitioning proof-of-concepts into production environments and therefore recommends that these gaps shall be addressed systematically. 

Among the largest gaps is the underestimation of infrastructure requirements \citep{Sharma2024, Hechler2020, Sundaramurthy2022}. The latest research states that the underestimation stems from computational demands in terms of the processing power required by AI models, the way AI models are served in terms of infrastructure setup and deployment expectations, and bottlenecks related to resource allocation when serving AI models to customers. 

The principle of cloud-native architecture with AI has been explored by several studies \citep{Shaban2025, Ali2018, Bhatia2025}. These studies often link easily accessible cloud-native architectures with fast deployment through APIs. By having a cloud-native architecture, the deployment cycles are shortened and therefore the AI services often reach better scalability and accessibility. 

When talking about regulated environments, a layered architecture is required \citep{Bhatia2025, Sundaramurthy2022}. This type of architecture is required due to the need for compliance and easier auditing capabilities. By having a layered architecture, it is easier to not only track changes but also provide customer excellence through better service levels.
 
The principles of resilience and security are often mentioned due to model attack vectors \citep{Sundaramurthy2022, Sharma2024, Hechler2020}. These studies often focus on the core principle of the SAIL framework being the scalability of AI systems. The key findings mention the importance of having graceful degradation frameworks as they are crucial for maintaining service levels in case of failures or attacks.

In order to ensure successful architecture and lotng-term sustainability, it is important to tailor AI architecture to the system as a whole \citep{Anny2024, Ali2018, Gupta2018}. These studies often outline their own frameworks, all agreeing on the principle of deeply embedding AI into the core of their architecture. This is opposed to the bolt-on approach as it reduces technical debt over a longer period of time. 

The need for a holistic orchestration framework has been outlined and thoroughly discussed in several studies \citep{Ali2018, Hechler2020, Shaban2025}. Studies often emphasize the importance of successful orchestration of various aspects, often noting DevOps principles, DataOps, and most importantly the cloud infrastructure. The evolution toward MLOps practices represents a critical maturation of AI deployment capabilities, providing structured approaches to operationalize machine learning at scale \citep{AWS2024, John2025}. 



\subsection{Human-Centric Organizational Transformation}

Another key part of the AI framework challenge is the human centric organizational transformation. There is a certain paradigm shift from technology centric to human centric approaches \citep{Fenwick2024, Wilson2023, Rinta-Kahila2022}. The recent publications are outlining that the human aspect has been heavily neglected due to the increased focus on technical aspects such as model accuracy, infrastructure, and data quality \citep{Fenwick2024, Wilson2023, Rinta-Kahila2022}. As majority of the research states, the largest reason for resistance in terms of AI, adoption are aspects of job security, the lack of skill and the cultural shift of the ever changing work environment. In order for a successful AI adoption the HR has to take step in place. The hour in the sense of Not only leading the entire change management process, but also creating a structured path for the engineers. This path has to not only focus on reskilling, but creating a culture of curiosity. and continuous learning 

The term collaborative intelligence comes into play when humans interact with AI Systems to augment their capabilities \citep{Wilson2023}. This principle is emphasizing on the complementary nature and the easy to use AI tools to elevate and enhance human capabilities rather than replacing them. Understanding the dynamics and determinants of technology adoption, as demonstrated in studies of blockchain and other emerging technologies, provides valuable insights that can be applied to AI implementation strategies \citep{Fosso2021}.

Several studies have tried to outline the basic principles of how to overcome the resistance to change when it comes to AI \citep{Rinta-Kahila2022, Chowdhury2022, Jarrahi2023}.
These studies came to a joint common finding, which is mentioning the importance of involving employees in the early process by setting clear goals and expectations and starting the process together as a company instead of being mandated to use certain AI tools. The company is able to win early trust and buy-in from the employees. 
What is often mentioned as a great way how to gain trust is to provide several pilot programs desmonstrating real value.
The studies are clearly showing that having a structured change management process always beats a strict mandated usage \citep{Rinta-Kahila2022, Chowdhury2022, Jarrahi2023}. 

Another important aspect to mention is the dilemma in between automation and augmentation \citep{Raisch2021, Brynjolfsson2023, Davenport2019}. This dilemma is outlining. The principle of AI adoption where some tasks are fully automated, while others are augmented. These studies all agree on the establishment of a healthy balance in between the two. 





\subsection{Strategic Frameworks and Organizational Readiness}

A very important aspect of any framework related to AI adoption must mention organizational readiness. Several frameworks to tackle this have already been established \citep{Kurup2022, Dasgupta2019, Radhakrishnan2022, Madanchian2025}. Some of the key findings from these frameworks and studies include the intersection of having a clear process for transformation along with ethical AI adoption. They emphasize that focusing on long-term sustainability can only be achieved when an organization is ready. These frameworks operate on the principle that a company needs to be properly evaluated, processes need to be established and measured in terms of the capacity of the company, and only then a certain change can be implemented.

Certain frameworks are outlining the process of nonlinear adoption and flexible framework principles based on experimentation, validation, and scaling. These are valid for the rapidly changing environment of tech \citep{Dasgupta2019, Radhakrishnan2022, Madanchian2025}. This type of approach is heavily emphasizing the strong importance of proof-of-concept validation. By being able to iterate, test, validate, and distribute to customers, the organization learns and becomes ready over time. The intersection between these studies is the core idea that a proper governance structure and a strong executive sponsorship lead to a higher rate of adoption.

Some studies are also outlining the principle of having a strong ethical integration \citep{Madanchian2025, Chowdhury2022, Jarrahi2023, Pumplun2019}. The ethical integration shall not be misled with the moral values of individuals. The principle of ethical integration in these studies is outlining that bias will be reduced with more data and more progress a company does. These studies are showing that with more data collected, bias is able to be reduced, and the rate of hallucination is correlated with hallucination in terms of the expectation from management.

As companies are moving towards more organizational readiness, a proper framework focused on risk management aligning with emerging regulatory requirements is needed \citep{Parlov2025}. This framework focuses on responsible AI deployment in the landscape of various regulatory challenges, such as the EU AI Act and ISO standards. The need to be compliant with the regulatory requirements in a given country is more important than ever before, with highly rising fees and penalties coming from governing bodies.

The last principle of organizational readiness comes with cost. Certain organizations try to push for radical change at times when the company is not ready for it, thus leading to a failed process of change management, wasted resources, and a decrease of trust from employees. As a company operates on the principle of a balance sheet, it is important to note that AI adoption is also working on the principle of cost as its main driver.

Several studies are noting that successful AI adoption can only yield a net positive outcome when the organization is ready for it \citep{Agrawal2022, Coombs2020, Berente2021, Makarius2020}. These studies are reporting on high productivity boosts and cost reduction with the use of AI automation augmentation, but mainly reskilling procedures.

\subsection{Software Sector and Technology-Native Organizations}

The software sector is the most relevant for the SAIL framework as it targets software organizations. Therefore, this section reviews already existing relevant frameworks that focus on AI adoption in software organizations and technology-native companies. A maturity model by \citep{Garousi2025} is a revolutionary model based on the principle of assessing maturity for a cohesive effect between individual and organization. This model focuses solely on the principle of AI maturity rather than just assessing the technical capabilities. 

The ability to combine both AI technical capabilities with organizational readiness showcases that it is possible to combine these two in order to create a cohesive system. It is very important to note that the model recognizes that software organizations require a tailored approach in terms of the assessment criteria as they often face unique challenges of integrating AI into their current processes. As each industry is different, it is harder to generalize throughout them. This framework specifically addresses software development companies from technical approaches such as machine learning models, automated testing, and pull requests. By highly focusing on the domain of technology, this model is very relevant.

In terms of generative AI adoption in software companies, a multiple case study was conducted \citep{Kemell2025}. This study focused on the same principle of moving from individual developer efforts into large organization orchestration. Throughout their research, it has been revealed that while many generative AI tools are providing significant value, it is necessary to approach this problem with a holistic framework encompassing reusable AI components, platforms, and managing technical debt, which often arises in AI-native environments. The conclusion of this research is that technology capabilities developed by individuals are often not sufficient to meet the expectations of management and can only be addressed when a strong framework comes into place and solves the question of how to develop AI software efficiently and scale it throughout the whole company. One of the key findings in this research is that AI product managers are becoming more and more relevant as they oversee the coordination between the data science teams, other management, and engineers. 



The question of scaling was also thoroughly examined \citep{Nwashili2025}. This research focused on developing a product management-focused framework which addressed the challenges related to scaling AI systems. As this study was conducted in a larger company, it is more relevant to companies with higher human capital. This framework provided actionable steps to ensure that multiple product teams have efficient collaboration. This research focused heavily on the principle of platform thinking and allowing reusable AI components and platforms. The core of this framework was an AI platform which allowed multiple teams to develop on top of it at a single time. This framework becomes very relevant especially due to its key findings, which once again highlighted the need to have efficient AI product managers. The most important result from this research is the need to deploy with a systematic approach while focusing on not duplicating efforts. 


Another AI-driven Enterprise Maturity Model has been developed and focused on the principle of utilizing maturity with progression \citep{Ern2025}. This maturity framework, specifically the AI-Driven Platform Enterprise Maturity Model (AIMM), once again mentions the platform principle \citep{Yablonsky2021}. It outlines that businesses should leverage AI as a core competitive advantage which would guide them on coordination of AI processes in a structured manner. Their framework focused on developing several platforms and entering these platforms once the teams become ready based on the internal evaluation metrics. This framework, however, emphasized deployment of real-time solutions due to the nature of the solution they were implementing. Therefore, the need to have appropriate maturity levels is very much justified. The key components of this research focused on AI automated testing and monitoring to ensure that the DevOps pipelines remain stable and the product meets the highest standards. This framework also mentions the Governance Board, which is a very important key factor in order to ensure that AI meets the compliance standards and does not violate any regulations and ethical principles. 


The last proposed framework, called the Multidimensional Hybrid Intelligence Framework, focuses on creating a comprehensive development environment which addresses the core problem of integration of humans and artificial intelligence \citep{Sherson2023}. 
The unique aspect of this paper is that it applies this framework to several key standards in terms of the human-AI pattern. The purpose of this framework was to solve the question of creating hybrid intelligent systems which are structured in order to effectively combine the interaction between humans and AI to solve and automate software systems. Throughout this framework, they chose a multidimensional approach which encompassed both technical architecture for human-AI collaboration and an interface so that there is seamless interaction between the two entities. This model emphasizes quality assurance. These quality assurance mechanisms focus on improving the output generated by AI via continuous improvements and distributed tasks between humans and AI systems. 
The key result from this framework is that it successfully recognizes that software organizations must move beyond the typical viewing of AI as a replacement of humans and rather create a multi-environment hybrid space allowing humans to interact with AI in a cohesive manner. The ability to complement and not compete is the core principle of this framework.





\section{AI Implementation and Scaling Frameworks: A Comprehensive Review}\label{ai-implementation-scaling-frameworks}

While the previous section examined general AI maturity models, this section focuses specifically on frameworks designed to guide the implementation and scaling of AI initiatives in organizations. Recent research has increasingly recognized that successful AI adoption requires more than maturity assessment---it demands structured approaches to navigate the complex journey from proof-of-concept to organization-wide deployment. This body of work directly informs the development of SAIL by highlighting critical success factors, common challenges, and proven strategies for AI implementation.

\subsection{Phases of AI Implementation}



The review conducted by \citep{Haefner2023} have came to to the conclusion that the successful implementation of AI requires a structured, phase-based approach that addresses both technical and organizational dimensions. Their conclusion of a three-phase model consists of firstly proving the concept, followed by productionizing, and finally platformizing AI capabilities. Their findings are highlighting the socio-technical nature of AI adoption and emphasizing on the fact that both technical and organizational factors must be addressed simultaneously for an AI adoption to be successful. Another very valuable insight from their research is the fact that competition in AI adoption is fierce and organizations that fail to implement structured frameworks risk being outcompeted by firms which are more innovative and focus on integrating AI into their operations with a holistc approach. This insight will be very valuable for the SAIL framework as the vision of it is to be based on the socio-technical approach as well as it not only involves technical capabilities of the engineers, but also the aspect of change management and the natural resistance to change, which every organization is and will be facing due to human nature.


In terms of the proving phase, the research focuses on demonstrating the technical feasibility throughout and actual controlled experiments. This process suggests that organizations will need to validate any of their AI-based solutions before comitting to any of them majorly in order for them to not waste valuable resources which could potentially drain the company. The strategic deployment of AI requires understanding both the technological possibilities and business value creation mechanisms \citep{Iansiti2020, Chui2018}. Organizations must identify high-impact use cases and systematically evaluate their potential before scaling \citep{Davenport2020}.




\subsection{Healthcare and High-Stakes AI Implementation}


Industry where AI is currently being used heavily and has a very positive impact on is healthcare. This industry involves strong regulatory requirements and very high stakes for patients lives. The study from \citep{Gama2022} focused on implementation of an AI framework into healthcare environment which very much differs from the software industry, but still provides valuable insights. Research has identified numerous barriers and facilitators specific to healthcare AI adoption, including technical infrastructure, clinical workflow integration, and regulatory compliance \citep{Hassan2024}. Additionally, marketplace dynamics and demand-side factors play crucial roles in healthcare AI adoption, requiring specialized frameworks that address the unique characteristics of healthcare ecosystems \citep{Singh2025}. The study outlines that trust with change adoption and major technological breakthrough tend to be mutually exclusive. Interesting insights from this study was the NASSS  (Non-adoption, Abandonment, Scale-up, Spread, and Sustainability) which emerged and tries to dig deeper into failing of AI technologies in the medical industry. 

As already mentioned that the software and healthcare very much differ the human capital stays the same. The responsible management of efficiently converting value to time stays the same. The study was mainly focusing on sustainability and scalability but not directly providing solutions. This will be the goal of the SAIL framework to show how framework can be still sustainable while scalable.



\subsection{Organizational Adoption Barriers and Enablers}

Several studies in the last few years have tried to examine the barriers which companies have to overcome in order to scale AI beyond the initial pilots. The latest exhibit being the study of \citep{Praveen2024} which came to the conclusion that many companies fail to convert pilot to solutions due to issues of integration complexity, scalability issues and once again the resistance to change. Research on AI adoption in public sector organizations reveals similar challenges, with implementation success depending heavily on organizational culture, leadership support, and change management strategies \citep{Chen2024}. This piece of work is very much emphasizing on the need to establish proper governance framework which fosters innovation and does not hurt motivation. The authors have found that successful AI transformation depends on creating an environment where experimentation is encouraged. This means that the failures are being treated as learning experiences rather than mistakes.

A case study from a very high-tech firm \citep{Tamburri2022} published before the rise of AI shows us that having a structured roadmap helps them convert the resistance to curiousity. They are also very much advocating for active flexibility due to the sudden changes in the industry and clear guidance from the side of the framework need. This study also focuses on having a strong feedback mechanism. One of the key pillars of SAIL framework and its structure which will be outlined in future chapters. The deployment and operation of machine learning systems present unique challenges that require careful planning and systematic approaches \citep{Baier2019}.


\subsection{Evidence-Based Approaches to Scalable Adoption}

An evidence-based framework specifically focused on bridging the gap between AI ambition and scaled impact addresses a critical problem: many organizations successfully complete AI pilots but struggle to systematically embed AI into operations at scale \citep{Pandiri2024}. The author identifies three key gaps that prevent successful scaling: the strategy gap (misalignment between AI initiatives and business objectives), the execution gap (lack of operational capabilities to deploy AI at scale), and the culture gap (organizational resistance and insufficient change management). SAIL directly addresses these gaps through its Strategic Alignment stage, Scale Deployment processes, and emphasis on change management throughout the lifecycle.

The evidence-based approach emphasizes the importance of clear success metrics, structured decision gates, and mechanisms for knowledge transfer across teams. Organizations that successfully scale AI typically establish centers of excellence, develop reusable components, and create communities of practice that facilitate learning and collaboration. These practices are incorporated into SAIL's governance structure and reuse mechanisms.

\subsection{Finance Sector and Domain-Specific Implementation}

The financial services sector presents unique challenges for AI adoption due to regulatory constraints, risk management requirements, and the need for explainable AI in high-stakes decision-making. A modeling framework specifically for scaling AI adoption in finance has been developed, utilizing multi-agent orchestration approaches \citep{Sepanosian2024}. Their implementation study demonstrated that domain-specific considerations must be integrated into AI frameworks rather than treated as afterthoughts. Financial institutions require robust audit trails, model validation processes, and mechanisms to ensure fairness and prevent bias---requirements that extend to other highly regulated industries.

While SAIL is designed as a general framework for software organizations, the finance sector research underscores the importance of flexibility and adaptability. Organizations must be able to customize the framework to their specific regulatory context, risk tolerance, and business requirements while maintaining the core structure that ensures coordinated and scalable adoption.

\subsection{Simplified Adoption Frameworks and Success Factors}

Recognizing that overly complex frameworks can themselves become barriers to adoption, research has worked toward a simplified AI adoption framework that identifies essential success factors for implementing AI-based information systems \citep{Kucevic2024}. Their design science research identified critical elements including executive sponsorship, cross-functional teams, iterative development processes, and clear value propositions. The authors argue that frameworks should provide sufficient structure to guide decision-making without imposing unnecessary bureaucracy that slows innovation.

This tension between structure and agility is central to SAIL's design philosophy. SAIL provides clear stages and decision gates while allowing organizations to iterate within stages, adjust timelines based on learning, and scale at a pace appropriate to their context. The framework is prescriptive enough to prevent common pitfalls but flexible enough to accommodate diverse organizational contexts and use cases.

\subsection{Guiding Frameworks for AI Readiness and Piloting}

A guiding framework focused on enabling successful AI adoption through careful assessment of organizational readiness has been developed \citep{Amling2024}. The framework emphasizes that readiness assessment is paramount before launching multiple AI initiatives simultaneously. Organizations must evaluate their data infrastructure, technical capabilities, talent availability, and cultural preparedness for AI adoption. The framework provides decision criteria for prioritizing AI use cases, determining optimal pilot scope, and establishing success metrics that balance technical performance with business impact.

This readiness-focused approach complements SAIL's Opportunity Scouting and Technical Feasibility stages, where organizations assess both the potential value of AI initiatives and their capacity to execute them successfully. SAIL extends this by providing explicit guidance on how to proceed when readiness gaps are identified---whether through capability building, external partnerships, or pilot scope adjustment.

\subsection{The Influence of AI on Firm Scaling}

Research exploring the relationship between AI implementation and firm scaling investigated how AI adoption influences organizational growth and scaling processes \citep{Mecca2024}. This research found that AI can both enable and constrain scaling, depending on how it is implemented. Organizations that successfully integrate AI into core business processes experience accelerated growth, improved efficiency, and enhanced competitive positioning. However, poorly planned AI implementations can create technical debt, organizational friction, and resource drains that actually impede scaling efforts.

This finding reinforces SAIL's emphasis on strategic alignment and architectural thinking from the outset. AI initiatives must be designed with scaling in mind from the beginning, rather than attempting to retrofit scalability after initial success. This requires considering factors like data architecture, model deployment infrastructure, and organizational change management as integral parts of the AI adoption process rather than secondary concerns.

\subsection{Synthesis: Implications for SAIL}

This review of AI implementation and scaling frameworks reveals several consistent themes that directly inform SAIL's design. First, successful AI adoption requires a phase-based approach that recognizes the distinct challenges of proving concepts, productionizing solutions, and platformizing capabilities. Second, socio-technical considerations are as important as technical ones---frameworks must address organizational culture, change management, and stakeholder engagement alongside technical architecture and data infrastructure. Third, governance mechanisms and clear decision gates are essential for managing risk and ensuring alignment with business objectives. Fourth, reuse and knowledge sharing across teams prevent duplication of effort and accelerate adoption. Finally, frameworks must balance structure with flexibility, providing clear guidance while allowing adaptation to organizational context.

SAIL integrates these insights into a comprehensive lifecycle framework specifically designed for software organizations. By combining the phase-based structure identified by Haefner et al., the readiness assessment approach of Amling, the barrier-focused perspective of Praveen et al., and the scaling emphasis of Pandiri and Mecca, SAIL offers a holistic approach to AI adoption that addresses both the technical and organizational dimensions of this complex challenge. The framework's eight-stage lifecycle, quality gates, and governance mechanisms provide the structure needed to coordinate AI initiatives across teams while maintaining the agility required in the fast-moving AI landscape.



\section{Synthesis and Identified
	Gaps}\label{synthesis-and-identified-gaps}


In order to properly synthesize the different theoretical lenses and existing model, the strategy to look at the gaps which are currently existing will need to be applied. No model or theory is perfect and in certain instances the existing models are simply not covering the needs and requirements of the modern software organizations and their solutions. The new wave of AI technologies and their adoption simply falls beyond the scope of the existing models and theories. There are currently several gaps in the literature and practice regarding AI adoption in software organizations and approaches which the software organizations are currently tackling.

In the next section we will outline the 5 most crucial gaps which are not only in the existing literature but also in the current practice of software organizations.

- Lack of lifecycle orientation. This results in the fact the current models which are existing do not provide a concrete approach which is covering the entire system lifecycle and thus not outlining the necessary steps from the early initial awareness until the late stages of full integration and being AI-Native.

- Insufficient focus on coordination across teams and departments, leading to fragmented AI initiatives:  In order for a model to be succesfully implemented and provide long term sustainable value, the coordination of AI initiatives across different teams and the knowledge transfer between them is beyong crucial as individual isolated efforts are not creating the desired value. The distributed nature of software organizations requires a structured approach to coordination. It is in fact natural for team to operate individually and focus on their own tasks, however from the perspective of achieving a common goal, this approach is not effective. Organizations must also address ethical implications and governance challenges throughout the AI lifecycle \citep{Vidgen2021, Felzmann2019}.

- Neglection of ``Reuse Mechanisms'': Very important aspect of AI adoption and efficiency on the way towards it is being able to develope AI assets (models, data, knowledge) which are reusable across various teams and projects. This type of initiative is able to unleash an enormous amount of efficiency and time savings as individual temas are not required to reinvent the wheel, but simply reuse existing assets and alter them to their specific needs. It is in fact expected that each team will be requiring a certain level of customization, but by ivnesting into a modular design and reuse mechanisms, the need to reinvent simply becomes obsolete.

- Holistic integration of theoretical lenses: There are certain individual theories as already mentioned in the previous chapters which are providing valuable insights and are well adopted in practice. However, the holistic approach of integrating these insights into a coherent solidified framework is missing. The theoretical lenses outlined are proven to work and are widely adopted in practice, thus the integration of these theories into a single framework is expected to provide a significant contribution to both theory and practice.


- Scalability challenges: Scalability is a crucial aspect of any software solution. It is extremely important to consider the question of "Does it scale?" when designing and implementing any type of solution which is meant to be use by wide audience. This aspect is beyong important when dealing with AI adoption as we are not only dealing with resistance of change but also with the complexity of AI technologies \citep{Keding2021}. If the solution implemented will not be able to scale and the first impression will be negative the chance to further invest into its development and reclaiming the trust of the organization will be extremely low. Therefore, scalability needs to be considered from the very beginning and the solution needs to be designed in a way which is able to scale across different teams and departments.



By addressing these gaps a synthesis strategy will be applied to look at the key components of each theory and model, looking at how will it overlap with the SAIL framework, where will it diverge in terms of approach towards AI adoption and finally which gaps will be addressed by SAIL in order to provide a clear rationale for its development



\begin{footnotesize}
\begin{longtable}{@{}
		>{\raggedright\arraybackslash}p{0.15\linewidth}
		>{\raggedright\arraybackslash}p{0.19\linewidth}
		>{\raggedright\arraybackslash}p{0.19\linewidth}
		>{\raggedright\arraybackslash}p{0.19\linewidth}
		>{\raggedright\arraybackslash}p{0.22\linewidth}
		@{}}
	\toprule
	\textbf{Theoretical Lens / Model} &
	\textbf{Key Components} &
	\textbf{Overlaps with SAIL} &
	\textbf{Divergences from SAIL} &
	\textbf{Identified Gaps Addressed by SAIL} \\
	\midrule
	\endhead
	\bottomrule
	\endlastfoot


\textbf{Enterprise Architecture (EA)} &
IT and business strategy alignment

Systemic integration

Governance

TOGAF and Zachman frameworks &
Blueprint for connecting AI to strategy

Governance mechanisms promote alignment &
Originates in pre-AI era

Lacks explicit lifecycle orientation

No reuse mechanisms &
Lifecycle focus

AI-specific governance

Asset reuse across projects \\
\midrule

\textbf{Digital Transformation (DT)} &
Digital-first orientation

Cultural and organizational readiness

Maturity models (Deloitte DMM) &
Phased maturity models

Emphasizes readiness as critical success factor &
Broad digital-change lens

Not specifically focused on AI challenges &
Aligns maturity thinking to AI lifecycle

Granular adoption stages beyond general digital readiness \\
\midrule

\textbf{Innovation Diffusion Theory (IDT)} &
Adoption curve (innovators $\rightarrow$ laggards)

Communication channels

Social systems &
Lens on AI adoption dynamics

Highlights roles of innovators and early adopters &
Focused on diffusion rather than organizational coordination or reuse &
Mechanisms for scaling

Coordination

Knowledge reuse across teams \\
\midrule

\textbf{Dynamic Capabilities Theory (DCT)} &
Sensing, seizing, transforming

Adaptability in volatile environments &
Agility lens for adopting and scaling AI &
No structured procedure for AI lifecycle &
Integrates sensing--seizing--transforming into AI lifecycle stages

Defined governance and processes \\
\midrule

\textbf{AI Maturity Models (Deloitte, PwC, Gartner, Forrester)} &
Step-by-step assessment of AI readiness across:

Strategy

Governance

Data

Talent &
Strong diagnostic tools

Starting points for AI adoption &
Static assessment focus

Lack lifecycle depth

No reuse strategies &
Actionable lifecycle roadmap

Supports adoption beyond maturity assessment into implementation \\


\end{longtable}
\end{footnotesize}


This comparative analysis clearly demonstrates the unique value
proposition of the SAIL framework. By integrating insights from multiple
theoretical lenses and addressing the specific gaps identified in
existing models, which SAIL aims to fill.

The gap of not having the structured approach to the entire AI adoption
lifecycle, from initial awareness to full integration is addressed by
SAIL through its clearly defined stages of adoption (Awareness → Pilot →
Scale → AI-Native).

\chapter{Theoretical Framework}\label{theoretical-framework}

The following chapter will outline the theoretical framework of SAIL.
Answering the questions like ``Why does SAIL exist?'', ``What
theoretical foundations is it built upon?'' and ``How do these theories
inform the design of the framework?''. The goal of this chapter is to
provide a clear understanding of the theoretical underpinnings of SAIL
and how are these theories being integrated into a cohesive framework.

\section{Conceptual Foundations of
	SAIL}\label{conceptual-foundations-of-sail}


In order to properly understand the conceptual foundations of SAIL, it
is extremely crucial to understand the purpose of a lifecycle adoption
framework. The main purpose of SAIL and any lifecycle adoption framework
is to provide a holistic and structured approach for a set of
activities, in this case the adoption of AI in a coordinated and
scalable manner. This aspect of lifecycle orientation is crucial as it
is not narrowly focused on a single aspect of the adoption process, but
looks on the landspace from an overarching perspective. The lifecycle
orientation can be understood as a series of stages that an organization
goes through in order to successfully adopt and integrate AI into its
operations while approaching the topic from a comprehensive perspective.




\subsection{Purpose of Scalable Artificial Intelligence
	Lifecycle
	(SAIL)}\label{purpose-of-Scalable-artificial-intelligence-lifecycle-sail}

The purpose of Scalable Artificial Intelligence Lifecycle (SAIL) is to
create a structured and systematic approach for software organizations
to adopt, scale, and integrate AI technologies effectively. SAIL is
aiming to serve as a roadmap that guides organizations through the
complex journey of AI adoption and integration, ensuring that AI
initiatives are aligned with business objectives, coordinated across
teams, and capable of delivering sustained value.

SAIL framework will not serve as a rigid prescription, but as a
practical playbook which will provide actionable steps for enterprises
to follow. It will not be a theoretical model. It will not be a true
``one-size-fits-all'' solution. It will not be an assesment model. It
will not be a maturity model. It will not be a diagnostic tool.

It will be a lifecycle-based roadmap outlining the stages of AI
adoption, from initial awareness to full integration and being
AI-Native, while staying true to solving the identified gaps and use
cases in the operational context of the organization.

Enterprises are currently standing in front of a major technological
shift, where AI is swiftly becoming a core component of business
strategy and operations. The ability to effectively exploit and leverage
this technology is becoming a key determinant of competitive advantage.

Efficiency is not secondary and innovation is not a luxury, but a
necessity for survival in the modern digital economy. Being second is
equivalent to being last.

\subsection{Positioning Relative to Digital Transformation and
	Enterprise
	Architecture}\label{positioning-relative-to-digital-transformation-and-enterprise-architecture}

While SAIL is conceived as an independent framework, it lies at the
confluence of Digital Transformation and Enterprise Architecture (EA).
To ensure that AI initiatives remain consistent with corporate
strategies and organizational objectives, SAIL must rest on the
principles of both Digital Transformation and EA.

Whereas companies will not transform themselves to fit the SAIL
framework, the framework itself must adapt to the organizational context
and structural realities. By applying EA principles, SAIL guarantees
that AI initiatives are embedded within the larger enterprise
architecture rather than developed in isolated silos. Since such global
alignment is indispensable, it provides the foundation for scalability
and acts as the enabler of coordination across organizational teams.

\subsection{Addressing Fragmentation through Coordination and
	Reuse}\label{addressing-fragmentation-through-coordination-and-reuse}

Even when organizations produce strong ideas or embrace advanced
technologies, such efforts inevitably fail without a clear coordination
mechanism. Because coordination is frequently overlooked prior to
innovation, promising concepts often remain unrealized, become
fragmented, or duplicate existing efforts, thereby undermining potential
synergies. By embedding coordination at the very outset of AI adoption,
organizations establish the conditions for effective integration and
sustainable scaling of AI initiatives.

Although AI adoption often appears to originate at the top, in reality
it begins at the operational level, where individuals experiment with
technology and emerge as its earliest champions. Only when senior
management creates the necessary environment and allocates appropriate
resources can these bottom-up innovations thrive. Since coordination and
information exchange serve as the glue of the adoption process, they
ensure alignment, integration, and synergy across organizational
activities.

The conceptual foundations of SAIL will be built upon theoretical lenses
which will be outlined in the following chapter. They will each provide
a beneficial and crucial part of the puzzle and will be carefully
integrated into the design of the framework.

\section{Integration of Theoretical
	Lenses}\label{integration-of-theoretical-lenses}


Integration as a business process is the act of coupling different
systems and processes together in order to seize value and synthesize a
final form of the product. This type of synthesis involves
cherry-picking the best parts of each individual component and linking
them together in order to compliment each other and create a more
valuable system.

SAIL framework will be built upon the integration of the theories
analysed in the chapter of Literature Review, where the analysis of the
existing literature and a brief introduction of the concepts was
conducted. The answer which needs to be provided is how these theories
support the design of the framework.

\subsubsection{EA and SAIL}\label{ea-and-sail}

Enterprise Architecture (EA) will be the backbone of the creation of the
SAIL framework. EA can be understood as the practice of analyzing,
designing, planning, and implementing enterprise analysis in order to
execute on business strategies. As SAIL framework is aiming to provide a
structured approach to the adoption of AI in organizations, the
similiarity of the outlined set of processe can not be overlooked. This
approach in enterprise architecture is crucial for not only setting up
initial organisation when going through the adoption process, but also
whenever enterprises are scaling or implementing new technologies or
processes. Regarding EA as already mentioned SAIL will focus more on the
model of TOGAF as it is more flexible and adaptable to the current
technological landscape and SAIL is expected to be more implementation
oriented and practical especially in the context of digital business and
companies which have started their digital transformation journey and
are lookging to implement AI in their operations. SAIL will need to be
aligned with the existing enterprise architecture of the organization
and most importantly will need to complement it. The principles of a
succesful EA will be the guiding principles for the design of SAIL and
will need to truly ensure that AI will not act as a disruptive force,
but rather as a true enabler of efficiency and innovation.

\subsubsection{Digital Transformation and
	SAIL}\label{digital-transformation-and-sail}

Digital Transformation consists of the integration of digital
technologies into all areas of a business, while fundamentally changing
how the business operates and delivers value to customers. Digital
Transformation is a major step for businesses which are not yet
connected to the digital world and are still operating in a traditional
way. Not every single business is required to undergo a digital
transformation, but for those who do, it is a major step which requires
a lot of resources and commitment. The root principle of Digital
Transformation is the ability to not only adapt and embrace new
technologies, but also to change the work culture in which the business
and the teams whithin the business operate. This might be a challenging
task, as it requires a lot of change management and the ability to
overcome resistance. The SAIL as a framework will have the presumption
that the organization is already digitally mature and has already
undergone a digital transformation. By targeting digitally mature
organizations, SAIL will be able to focus on the actual adoption of AI
and not on the prerequisites which are required for a successful
adoption. This will lead to efficient and streamlined internal process
rollout and not wasting resources on the prerequisites. This also rules
out the risk of failure due to lack of readiness and thus increases the
chances of success. The principles of Digital Transformation are still
going to hand in hand with SAIL, as SAIL will not be acting as a digital
transformation framework, but rather as an AI transformation framework
which is complementing the existing digital transformation efforts of
the organization.

\subsubsection{Innovation Diffusion Theory and
	SAIL}\label{innovation-diffusion-theory-and-sail}

Innovation Diffusion Theory (IDT) is a theoretical framework which
explains how, why, and at what rate new ideas and technology spread
through cultures. The ideology underlying IDT and SAIL is innovation and
the adoption of new technologies. The principles of IDT will be crucial
for the design of SAIL, as it will provide a clear understanding of how
AI adoption is happening in organizations and what are the key factors
which influence the adoption process. From early adopters to laggards,
the principles of IDT will be taken into consideration when discussing
and implementing the collaboration and coordination mechanisms of SAIL.
The principles of communication channels, social systems, and time will
be crucial for the design of SAIL and its stages of adoption will be
inspired by the IDT framework. SAIL will encompass the principles of IDT
regarding reuse and knowledge transfer between teams and will provide a
clear roadmap for the adoption process. IDT supports SAIL as SAIL is
based on innovation and is one of its core principles. The innovation
aspect in SAIL is the use case driven approach, where the adoption is
driven by the actual use cases and their value proposition. This is a
crucial aspect as it ensures that the adoption is not happening in a
vacuum, but is driven by the actual needs of the organization.

\subsubsection{Dynamic Capabilities Theory and
	SAIL}\label{dynamic-capabilities-theory-and-sail}

Dynamic Capabilities Theory (DCT) is a theoretical framework which is
explaining how organizations can adapt and thrive in rapidly changing
environments. DCT and SAIL are closely related as SAIL will integrate
the 3 main components of DCT into its design and those being the
sensing, seizing, and transforming. SAIL will be starting with the
sensing of opportunities, where the organization will need to perform a
short analysis and audit on the gaps which they will be need to fill in
order to successfully adopt AI. This will be followed by the seizing of
opportunities and prepare a framework which will serve as a benchmark
for the actual adoption process. The final step will be the transforming
of the organization, where efficient communication and implementation of
the framework will be crucial for the success. The principles of DCT are
in fact very crucial and relevant for the SAIL framework as they will
inspire the design of the framework and SAIL will build upon the
principles of DCT in order to provide a structured approach to the
adoption process. The ability to sense, seize, and transform will be
crucial for the success of the adoption process and will be the guiding
principles for the design of SAIL.




In order to properly summarize the process of integration of theoretical lenses and their impact on the design of SAIL, the following table will outline each of the theories, their key concepts, design implications, and how they are being implemented in SAIL.


\begin{table}[htbp]
\centering
\caption{Integration of Theoretical Lenses into the SAIL Framework}
\label{tab:theoretical_integration_sail}
\begin{tabular}{p{3.2cm} p{3.2cm} p{4.2cm} p{4.2cm}}
\hline
\textbf{Theory} & \textbf{Key Concept} & \textbf{Design Implication} & \textbf{Implementation in SAIL} \\
\hline
Enterprise Architecture (EA) &
Business--IT alignment, governance, systemic integration &
AI initiatives must align with organizational strategy and architectural principles &
Strategic alignment checks at each SAIL stage; defined governance roles and architectural consistency across AI initiatives \\

Digital Transformation (DT) &
Organizational readiness, digital maturity, cultural change &
AI adoption requires phased organizational enablement beyond technology alone &
Lifecycle progression with readiness considerations embedded in early SAIL stages \\

Innovation Diffusion Theory (IDT) &
Adoption stages, role of innovators and early adopters &
AI adoption progresses unevenly across teams and requires coordination mechanisms &
Lifecycle stages supporting experimentation, validation, and structured scaling across organizational units \\

Dynamic Capabilities Theory (DCT) &
Sensing, seizing, transforming &
AI adoption must remain adaptive in dynamic and uncertain environments &
Iterative feedback loops and decision gates enabling reassessment and transformation across SAIL stages \\

AI Maturity Models &
Diagnostic assessment of AI readiness &
Readiness assessments must translate into actionable implementation guidance &
SAIL extends maturity diagnostics into an operational, end-to-end AI lifecycle \\
\hline
\end{tabular}
\end{table}


\section{Principles Derived for Framework
	Design}\label{principles-derived-for-framework-design}

The outcome of the synthesis of the theoretical lenses and the
conceptual foundations of SAIL will be the derivation of the key
principles which will guide the design of the framework. These
principles will be crucial for the success of the adoption process and
will ensure that the framework is aligned with the theoretical
underpinnings. The key principles which will guide the design of SAIL
are:

\begin{itemize}
	\tightlist
	\item
	Scalability: Scalability acts as a principle of the framework, as it
	will ensure that the adoption process is not happening in a vacuum and
	will be able to outgrow the initial adopters working on the AI
	solution during their pilot phase. The ability to scale the adoption
	process across different teams and departments will be crucial for the
	success of the adoption process and will ensure that the AI
	initiatives are able to deliver sustained value. Scalability will be
	achieved through the use of coordination mechanisms and reuse
	mechanisms which will be outlined in the following principles.
	\item
	Coordination: Coordination as an actor in the process has been already
	outlined in the previous chapter and will be crucial for the success
	of the adoption process as without a clear communication no message
	will be able to be delivered and the adoption process will fail with
	certainity. This polemical message is crucial for the design of SAIL
	and insisting on the importance of coordination will be the one of the
	most detrimental aspects of the framework.
	\item
	Value-driven use case prioritization: The adoption process will be
	driven by the actual use cases and their value proposition. This will
	be achieved by the initial DCT analysis which will be performed at the
	beginning of the adoption process by a small team of experts. The use
	cases will be prioritized based on their feasibility, impact, and
	alignment with the overall business strategy. This will ensure that
	the adoption process is not happening in a vacuum and is driven by the
	actual needs of the organization. The early adopters acting as the
	champions of the adoption process will be crucial for the success of
	the adoption process and will ensure that the AI initiatives are able
	to deliver sustained value.
	\item
	Mechanisms for capturing and reusing AI assets: The ability to capture
	and reuse AI assets (models, data, knowledge) remains as an important
	factor of the framework and the ability to repurpose AI initiatives
	and being able to branch from the initial use case will be very
	beneficial for the success of the adoption process as the teams will
	be able to save time and solve multiple usecase withe the same AI
	asset. This will ensure that the adoption process is not happening in
	a vacuum and the AI asset will be able to deliver value accros
	multiple divisions and departments. The reuse mechanisms will be
	crucial for the success of the adoption process and will ensure that
	the AI initiatives are able to deliver sustained value.
\end{itemize}

\chapter{Research Methodology}\label{research-methodology}

The upcoming chapter will focus on the research methodology which will
be used for the development of the SAIL framework. This chapter will
outline the research design, literature review approach, framework
development process, and conceptual evaluation via use cases. Research
methodology is a crucial aspect of any research project, as it is going
to define and establish the credibility and validity of the research
findings. The chosen methodology will be aligned with the research
question and objectives, ensuring that the research is able to provide a
solid an comprehensive answer to the research question of ``How can
software organizations effectively adopt and scale AI technologies
through a structured lifecycle framework?''.

\section{Research Design -- Design Science
	Research}\label{research-design-design-science-research}

This thesis is using the model called Design Science Research (DSR) as
the research methodology. DSR is a research paradigm which is focused on
the creation and evaluation of artifacts (models, methods, frameworks)
in order to solve real-world problems. This thesis is focused on the
creation of a conceptual framework (SAIL) that solves a real-world
problem (AI adoption in software organizations). The DSR methodology is
particularly well-suited for this research as it provides a structured
approach for the development and evaluation of the framework. The target
of this thesis is not literature review, nor is it a metanalysis of
existing frameworks. The primary and sole goal of this thesis is the
resugence of a new framework which is able to solve the identified gaps
and use cases in the operational context of the organization.

The DSR methodology consists of three main cycles: relevance cycle,
rigor cycle, and design cycle. The relevance cycle is focused on the
identification of the problem and the context in which the problem
exists. The rigor cycle is primarily focused on the review of existing
literature and theories in order to provide a theoretical foundation for
the framework that is being developed. The last stage of the DSR
methodology is the design cycle, which is focused on the actual
development and evaluation of the framework. The last stage will be the
most important one, as the actual process of pitching and outlining the
framework will take place. The evaluation of the framework will be done
through the use of realistic use cases, which will be outlined in the
following chapter. The use cases will be used to evaluate the framework
against multiple criteria, such as coherence, scalability, and
reusability. The use cases will be selected based on their relevance to
the research question and their ability to provide a comprehensive
evaluation of the framework. The use cases will be based on realistic
scenarios that software organizations might face when adopting and
scaling AI technologies.

To adhere to the DSR methodology, the research will be conducted in a
structured manner, following the three cycles outlined above.

\subsection{Relevance Cycle}\label{relevance-cycle}

The relevance cycle will be focused on the identification of the problem
and the context in which the problem exists. The problem of AI adoption
in software organizations has been identified as a real-world problem
that needs to be addressed. The context in which the problem exists is
the software industry, which is rapidly evolving and adopting new
technologies. The relevance cycle will also involve the identification
of the stakeholders who are affected by the problem, such as CTOs,
innovation managers, and engineering leads. The relevance cycle will
provide a clear understanding of the problem and its context, which will
inform the development of the framework. Relavance as a cycle will be
outlined in the SAIL framework as the first stage of the adoption
process, where the organization will need to perform a short analysis
and audit on the gaps which they will be need to fill in order to
successfully adopt AI. This will prove whether the organization is ready
to adopt AI and will provide a clear understanding of the problem and
its context, which will inform the development of the framework and
therefore the relevance cycle is being integrated into the design of
SAIL.

Connection of Relevance cycle to SAIL: - SAIL is addressing
fragmentation through coordination and reuse, which is a real-world
problem that needs to be addressed. - The Relevance cycle emphasizes on
the importance of understanding the context in which the problem exists,
and on the practical needs - Use cases will act as a bridge between
theory and practice, ensuring that the framework is relevant to the
needs of software organizations.

The Relevance Cycle guided SAIL's development, by being able to identify
the aoption challanges while ensuring that the framework is addressing
the practical needs of software organizations.

\subsection{Rigor Cycle}\label{rigor-cycle}

The goal of the rigor cycle is to provide a solid theoretical foundation
for the framework that is being developed. The rigor cycle is involving
the review of existing literature which was done in the chapter of
theoretical frameworks where a thorough analysis of the existing
literature and theories was conducted and its connection was established
and the gaps were identified. By being able to connect the existing
theories to the design of the framework, the rigor cycle increases on
the validty as the framework is not being developed in a vacuum, but is
based on solid theoretical foundations. Rigor cycle basis its
foundational principles on the integration of multiple theoretical
scopes and lenses into one meta-model which serves as the backbone of
the framework. The principles derived from the synthesis of the
theoretical lenses will be crucial for the design of SAIL, as they will
provide a clear understanding of how to leverage the internal resources
of the organization in order to successfully adopt AI and create value.

Connection of Rigor cycle to SAIL: - Chapter of theoretical frameworks
provided a solid theoretical foundation for the framework that is being
developed. - The listed theories of EA, Digital Transformation, IDT and
DCT serve as information sources for the design of SAIL. -
Ensures that SAIL is not arbitrary but is grounded in established
knowledge.

The Rigor Cycle ensured that SAIL is grounded in established knowledge,
by providing a solid theoretical foundation for the framework that is
being developed.

\subsection{Design Cycle}\label{design-cycle}

The design cycle or otherwise known as the build and evaluate cycle will
be the cycle which lays its scope on the foundational principles of the
SAIL framework. The design cycle is focused on the actual development
and evaluation of the framework. The primary focus of this thesis is the
development of the framework, which will be done in a structured manner,
in the upcoming chapter. The evaluation of the framework will be done
through the use of realistic use cases where each use case will be
evaluated against multiple criteria and will be specified on how to
overcome it in a structured manner. The use cases will be selected based
on their relevance and the possibility of them occuring in a real-world
scenario in the ambit of software organizations. The foundation of the
model will be based on the principles derived from the synthesis of the
theoretical lenses, however as it is a new framework, it would be
unprofessional to limit the level of innovation and creativity by simple
observing the existing theories. SAIL as a framework will be innovative
and creative in its design, while still being grounded in established
knowledge. While the theoretical lenses provide the foundation, the
novel contributions of SAIL are a product of independent thought and
design. This balance between theory and innovation is crucial for the
success of the framework, as it ensures that the framework is both
relevant and practical. By pushing the boundaries of existing knowledge,
SAIL aims to provide a new and breakthrough approach to AI adoption in
software organizations.

Connection of Design cycle to SAIL: - The design cycle is focused on the
actual development and evaluation of the SAIL framework which is the
primary focus of this thesis. - Use cases will be used in the evaluation
process of the framework, ensuring that the framework is practical and
relevant. - While grounded in theory, SAIL incorporates innovative
elements that extend beyond existing models.

The Design Cycle facilitated the creation of SAIL. It is inevitable that
the design of SAIL will be innovative and creative, while still being
grounded in established knowledge and the process of outlining followed
by the evaluation of the framework will be done through the use of
realistic use cases.

\section{Literature Review
	Approach}\label{literature-review-approach}

The question of why each of the theories was chosen and how they are
connected to the design of SAIL was already answered in the chapter of
theoretical frameworks, however in order to provide a clear
understanding of the literature review approach, the following segment
will elaborate on the search strategy, inclusion/exclusion criteria, and
the process of synthesizing theory.

\subsection{Search Strategy, Databases,
	Keywords}\label{search-strategy-databases-keywords}

The search strategy for the literature review was focused on identifying
relevant articles, models, and frameworks that are addressing the topic
of AI adoption in software organizations, which is the primary focus of
this thesis. The search was done over multiple databases, including
Google Scholar, IEEE Xplore, ACM Digital Library. These databases were
helpful in providing a wide range of articles and models from both
academic and industry sources. The search was not only limited to
academic articles, but also included industry reports and whitepapers
from leading consulting firms such as Deloitte, PwC, Gartner, and
Forrester.

The problematic aspect of AI adoption as already mentioned is a very
recent topic and therefore the most relevant and up-to-date information
can be found either by direct reports of large consulting firms or in
conference papers and articles or journals which are focusing on the
topic of AI adoption and its challenges.

The keywords used in the search included combinations of terms such as
``AI adoption'', ``Artificial Intelligence implementation'', ``Digital
Transformation'' and other outlined frameworks. Regarding the research
on the already existing frameworks, the process was more straightforward
as the frameworks are already established and therefore the search was
more focused on finding the most relevant and up-to-date information
about the frameworks and their principles.

\subsection{Inclusion/Exclusion
	Criteria}\label{inclusionexclusion-criteria}

The inclusion criteria for selecting articles and models were based on
their relevance to the research question, publication date (preferably
within the last 10 years to ensure contemporary relevance), however this
was not a strict rule as some of the theories are older but still
relevant such as the IDT. Inclusion of the articles was also based on
their citation count, with a preference for highly cited works which
indicates their influence and acceptance not only in the academic
community but also in the industry.

\subsection{Process of Synthesizing
	Theory}\label{process-of-synthesizing-theory}

The process of synthesizing theory through the literature review
involved a thorough analysis of the selected articles and models. The
main motivation behind the synthesis was to identify the key principles
and concepts that are relevant which are relevant in the context of AI
adoption in software organizations. The main contribution of the
synthesis was the identification of the gaps and the ideation process
started with the cardinal question of ``Where do existing frameworks
fall short in addressing the challenges of AI adoption in software
organizations?'' This question was crucial for the design of SAIL, and
the frameworks which are crucial and are the cornerstone of the design
of SAIL. There were many more frameworks which were analysed, but the
ones which were chosen are the ones which are most relevant and are
truly the depiction of the vision of the SAIL framework. Frameworks
which were not chosen were either too generic and did not provide any
specific insights into the topic of AI adoption or were too specific and
did not provide a holistic view of the adoption process. In the end the
list of the theories which were chosen are the ones which are most
relevant and are truly the foundation of the design of SAIL, both from
the perspective of theory and practice.

\section{Framework Development
	Process}\label{framework-development-process}


The framework development process will be based on an iterative design
and refinement approach. Multiple iterations of this framework were done
in order to ensure that the framework is practical and relevant and the
final version of the framework is the one which is able to solve the
identified gaps and use cases in the operational context of the
organization, which is the one that will be presented in the upcoming
chapter. This process of the development and the refinement of the
framework was done in a structured manner, following the principles
derived from the synthesis of the theoretical lenses. Each of the
theoretical lenses defined in the chapter 3 will need to be connected to
a realisitc part of th SAIL framework, otherwise the point of the
chapter 3 is simply lost. The following segment will outline the mapping
of the theories to practical framework elements.

\begin{itemize}
	\tightlist
	\item
	Enterprise Architecture (EA) → systemic integration: EA principles are
	guiding the design of SAIL to ensure that AI initiatives are embedded
	within the larger enterprise architecture rather than developed in
	isolated silos. The improtance of this theoretical framework will play
	part on effective integration and sustainable scaling of AI
	initiatives.
	\item
	Digital Transformation → readiness and strategic alignment: SAIL
	presumes that the organization which shall adopt the framework is
	already digitally mature and has already undergone a digital
	transformation. This leads to efficient and streamlined internal
	process rollout and not wasting resources on the prerequisites, such
	as data infrastructure and digital culture.
	\item
	Innovation Diffusion Theory (IDT) → adoption dynamics across teams:
	IDT does not only serve as the amplifier of the innovation aspect of
	SAIL and the use case driven approach, but still gives the framework a
	clear understanding of how AI adoption is happening.
	\item
	Dynamic Capabilities Theory (DCT) → agility in scaling and
	transformation: The principles of sensing, seizing, and transforming
	are guiding the design of SAIL and are integrated into the stages of
	adoption. The ability to sense, seize, and transform will be crucial
	for the success of the adoption process and will be the guiding
	principles for the design of SAIL.
\end{itemize}

Each of these theories is a unique piece of the puzzle and when combined
together they are able to create a ruleset and outline the scope of the
SAIL and give an estimate on the ability of the framework to solve the
identified gaps and use cases in the operational context of the
organization.

\section{Conceptual Evaluation via Use
	Cases}\label{conceptual-evaluation-via-use-cases}

Extremely crucial part of the DSR methodology is the evaluation of the
developed framework. Great idea is only as great as the strongest
mechanism which is validating it. The larger and harsher the validation
mechanism is, the more credible and valid the idea becomes. By testing
the framework against realistic use cases which are not only focused on
extreme scenarios, but also on the more common ones, the framework is
able to prove its versatility and applicability in a wide range of
scenarios. The framework will most likely not be able to solve every
single use case which is thrown at it due to extremely large amount of
unknown and unpredictable factors which are influencing the adoption
process. However, by trying to solve a wide range of use cases, which
occur at the companies will give a clear understanding of the strengths
and weaknesses of the framework and will provide a solid foundation for
future research and development of the framework. The interpretation of
the results will be crucial for the success of the framework, as it will
provide a clear understanding of the strengths and weaknesses of the
framework and will inform future research and development of the
framework. The outcomes will need to be interpreted in a structured
manner with the ability to reduce a bias and provide a clear
understanding of the strengths and weaknesses of the framework. By
having a great suite of testing scenarios, the framework is able to
prove its versatility and applicability in a wide range of scenarios.
The choice of usecases might be even more crucial than the actual design
of the framework, as the use cases will be the ones which will validate
the framework and will provide a clear understanding of its strengths
and weaknesses. Seeing from which aspect the framework is outperforming
and from which aspect it is underperforming will be crucial for the
future development of the framework and will provide a solid foundation
for future research and development of the framework and its adoption in
the real-world scenarios.

\chapter{Framework Development
	(SAIL)}\label{framework-development-sail}

\section{Overview of SAIL Framework}\label{overview-of-sail-framework}

The Structured AI Lifecycle (SAIL) framework is a nexus between the
theoretical foundations of AI adoption and the practical realities faced
by software organizations. SAIL is designed to provide a structured
approach of company-wide AI adoption, ensuring that business value is
maximized while being focused on the operational aspect of the business.
By building on the theoretical foundations outlined in chapter 3 and
methodological rigor from chapter 4, SAIL operates in a manner of
defining clear design principles and practical steps such as
scalability, coordination, value-driven use case prioritization, reuse
mechanisms, and governance structures.

SAIL is structured into eight-step lifecycle stages, each addressing a
critical phase of AI adoption. The AI adoption journey encompasses
stages from early initial opportunity identification to full AI-native
integration and optiomization. Each stage is designed to solve a
specific set of challenges that organizations typically face during
their AI adoption journey. None of these stages exist in isolation;
rather, they are interconnected and their combined effect is greater
than the sum of their parts. The butterfly effect of each stage is
crucial as individual decisions on stage level amplify the overall
success of the AI adoption journey.

The north star of SAIL is to ensure that AI initiatives are not only
successfully launched but are also scalable, coherent, and reusable
across the organization. Furthermore SAIL will be presented in a visual
diagram which will outline the different stages of the framework and
their interconnections. The visual representation will provide a clear
understanding of the framework and will serve as a reference point for
the implementation of the framework in real-world scenarios. The SAIL
framework is iterative and flexible, allowing organizations to adapt it
to their specific contexts while adhering to the core principles of
effective AI adoption and therefore the need to have edge cases and
deviations from the outlined path is inevitable.

SAIL framework will be evaluated through the use of 10 realistic use
which are going to be outlined in the upcoming chapter. This will serve
as the empirical part of the DSR methodology and will provide the
necessary validation for the framework.

This chapter will outline the SAIL framework in detail, describing each
stage, its objectives, key activities, roles involved, decision points,
and expected outputs. This chapter will not only propose the framework
but will also dig into the practicalities of its implementation,
governance and knowledge reuse mechanisms. The goal is to provide a
comprehensive guide that organizations can follow to navigate the
complexities of AI adoption effectively.


\section{Lifecycle at a Glance}\label{lifecycle-at-a-glance}

\begin{itemize}
	\item
	Use a diagram here (8-step lifecycle with feedback loop).
\end{itemize}

At a high level, the SAIL framework consists of the following eight
stages: 1. \textbf{Opportunity Scouting:} Identifies and prioritizes
high-impact AI opportunities where the impact of AI can be maximized
while the effort to implement is minimized. This stage involves
planning, idea collection, initial value estimation and effort
estimation. Effort vs impact matrix will be used to prioritize the use
cases. 2. \textbf{Strategic Alignment Review:} Ensures that selected
opportunities align with business strategy and have stakeholder buy-in.
This stage involves stakeholder review, alignment workshops, and
strategic fit analysis, which are crucial for the success of the
adoption process. 3. \textbf{Feasibility \& Risk Assessment:} Assesses
technical debt, organizational readiness, and risks which are inherent
to the adoption process. The objective of this stage will be to identife
whether the use case is feasible and whether the risks are manageable.
4. \textbf{Pilot Design:} Plans the pilot implementation, defining
scope, success metrics, and resource allocation. The aim of this stage
is to ensure that the pilot is well-defined and has clear success
criteria. Throughout this stage there is not a single pilot being
designed. In ordert to be as efficient as possible, multiple pilots are
being designed solving different use cases. This will ensure that the
organisation is able to diversify its AI portfolio and is not putting
all its eggs in one basket. 5. \textbf{Pilot Execution:} Builds and
tests the AI solution in a controlled environment, collecting user
feedback and iterating as needed. The purpose here is that the pilot is
able to solve and deliver painpoints outlined in the initial stages. The
pilots will need to be evaluated against the success criteria outlined
in the planning stage. Focus on quick wins and fast iterations to build
momentum is the key to success. 6. \textbf{Evaluation \& Scaling
	Decision:} Evaluates pilot outcomes against success criteria to decide
on scaling, pivoting, or terminating the initiative. This stage serves
as a make it or break it point for the adoption process. Throughout this
phase the decision of whether to scale the pilot, iterate on it, or
abandon it will be made. The decision shall not be taken lightly, and
the emotions shall be taken out of the equation. The only thing that
matters is the value delivered and the potential for future value. 7.
\textbf{Scale Deployment:} Rolls out the solution organization-wide,
managing change, training, and integration with existing systems. The
objective of this stage is to ensure that the solution is successfully
scaled across the organization and is able to deliver sustained value.
Change management and training will be crucial for the success of this
stage. 8. \textbf{AI Native Optimization:} Integrates, optimizes, and
continuously improves the AI solution, embedding it into business
processes and culture. The final stage of the adoption process will be
the optimization and continuous improvement of the AI solution. The
objective of this stage is to ensure that the AI solution is able to
deliver sustained value and is integrated into the business processes
and culture of the organization.

The outlined set of stages is designed to be flexible and adaptable,
allowing organizations to tailor the framework to their specific
contexts while adhering to core principles of effective AI adoption. The
driver force of rapid innovation and value-based delivery principle
enhances the long-term success of AI initiatives. In order to understand
the framework in a more structured manner, a diagram will be presented
to illustrate the stages, their interconnections, and the overarching
governance structure.

{[}Insert Diagram Here: 8-step lifecycle with feedback loop{]}

\section{Governance Layer Across
	Stages}\label{governance-layer-across-stages}

Governance as a concept is inevitable for the success of the adoption
process and will be embedded across all SAIL stages, ensuring oversight,
alignment, and conflict resolution. Governance is a term often
misunderstood and misused, but in the context of SAIL, governance is not
about bureaucracy or red tape. It is about ensuring that the adoption
process is aligned with the overall business strategy and that the key
decision points are made in a structured and transparent manner. If the
governance mechanisms are not in place, the adoption process has an
increased chance of failure due to the simple principle of not following
a structured approach. The successful governance mechanisms encompasses
several key stakeholders and roles with each having a clear
understanding of their responsibilities and decision-making authority.
The key roles involved in the governance of SAIL include: -
\textbf{Sponsor:} Champions the AI adoption initiative. Their
responsibilities include securing resources, advocating for the
initiative at the executive level, and ensuring alignment with strategic
objectives. These sponspors are unique individuals who are able to see
the big picture and are able to drive the adoption process forward.
Their crucial role in the process is the ability to secure resources and
advocate for the initiative at the executive level. Their work is mostly
seen in the initial stages of the adoption process, but their influence
is felt throughout the entire lifecycle. They are usually the
individuals who are well spoken, value oriented and are able to present
and sell the idea of AI adoption in a structured manner. -
\textbf{Governance Board:} A cross-functional team, being responsible
for approving/rejecting use cases. Often might be seen as a bottleneck,
but their role is crucial for the success of the adoption process. Their
key responsibilities include reviewing use cases for strategic fit,
resolving conflicts (e.g., overlapping pilots), and ensuring that the
adoption process adheres to organizational policies and standards. This
board shall be composed of senior analysts and engineers together with
business stakeholders who are able to provide a holistic view of the
adoption process and value delievery. Senior analysits providing the
analytical perspective and potential bottlenecks, opportunities and
directions for the AI solution while the business stakeholders are able
to provide the business perspective and roadmap allignment with the
overall business strategy. - \textbf{Adoption Manager:} Oversees the
day-to-day execution of the SAIL framework. Their responsibilities
include coordinating activities across stages, managing timelines, and
reporting progress not only to the sponsor but also to the governance
board and other stakeholders such as team leads and managers. The
adoption manager is the glue holding the entire adoption process
together, while ensuring that deadlines are met and that the
communication is clear and transparent. The connection between sponsor
and adoption manager will be a holding point as they are both focused on
the success of the adoption process, but from different perspectives.
The sponsor is focused on the strategic aspect while the adoption
manager is focused on the operational aspect. - \textbf{Teams/Pilot
	Teams:} Execute tasks within each stage, providing feedback and
insights. Their primary responsibility is the implementation of the AI
solution outlined. They are not only the key to the value creation but
also the key to the success of the innovation and idea generation. Teams
implement the ideas throughout their day to day work and are the ones
creating the PoCs. The very important aspect of the teams will also be
in the idea generation process. In a software companies engineers are
the majority of the workforce and vale creation is happening through
their work. By letting engineers draft and propose usecases which are
optimally company wide painpoints, the AI adoption process is able to
solve painpoints which are actually existing and are not just
theoretical and have an impact on the day to day work of all engineers.
Engineers are the ones who are able to see the painpoints and are able
to propose solutions which are actually solving internal bottlenecks and
by letting the pilot team try to solve these painpoints with AI. The
overall company is able to yield the benefits of the AI adoption process
and is able to deliver same value in shorter timeframes allowing for
more innovation and creativity. - \textbf{Team Leads/Managers:}
Facilitate team involvement, ensure resource availability, and support
change management during scaling. The Team Leads and Managers play a
very important role in the adoption process and their integration in it
is pivotal for the success of the adoption process. The Sponsor is
negotiating team capacity for the pilot teams and their engineers as the
engineers are not expected to work on AI initiatives full time, but
rather as a side project. The Team Leads and Managers are the ones who
need to adhere and ensure that the capacity is available and that the
engineers are able to dedicate time to the AI initiatives. Without the
support of the Team Leads and Managers, the adoption process will fail
as engineers will be forces to work on the AI initiatives outside of
their working scope and the overall quality of the work will be
diminished. The Team Leads and Managers are the ones who are able to
provide the necessary support and resources for the success of the
adoption process.

\section{Key Decision
	Checkpoints}\label{key-decision-checkpoints}

As already mentioned earlier, the SAIL framework incorporates key
decision checkpoints and works with dynamic events and processes which
are influencing the adoption process. There is a set of key decision
checkpoints which are crucial for adoption process and will serve to
determine whether the initiative is viable and should proceed to the
next stage. After each of the key decision checkpoints, there is a
possibility to either proceed to the next stage, iterate on the current
stage, or abandon the initiative altogether. Without successful
governance mechanisms, these decision points can become bottlenecks or
sources of conflict, hindering progress and diluting value. Every single
one of the stages has its own decision points and shall only be
proceeded to the next stage if the decision points are successfully
passed. By having clear decision points, the teams and the organisation
are able to reduce the risk of failure, track the progress are receive
feedback on the progress of the adoption process not only from
customer's persective but also from the internal stakeholders. As
already very briefly mentioned, each of the stages is sequential but
iterative, meaning that the stages are not linear and there is a
possibility to go back to previous stages if the need arises. Before
entering the next stage, the decision points need to be successfully
passed and a question of whether to proceed, iterate, or abandon needs
to be answered. The 3 possible outcomes of each decision point are: -
\textbf{Proceed:} This initiative meets the strategic, technical, and
operational criteria in order for the team to move to the next stage.
The proceed option can only be chosen if the initiative is able to
deliver value and is aligned with the overall business strategy. The
proceed option is the most desirable outcome as it indicates that the
initiative is on track and is able to deliver value. - \textbf{Iterate:}
The initiative shows promise but requires further refinement. Either in
the form of additional research, pilot adjustments, or stakeholder
engagement. The iterate option is chosen when the initiative is not yet
ready for it to be proceeded into the next stage. The ability to iterate
is crucial as it involves dynamic requirements and does not accept
mediocrity. The option to iterate involes a 2 man's game where the
manager and the team are able to work together in order to refine the
initiative and make it ready for the next stage. - \textbf{Abandon:} The
initiative does not meet the necessary criteria and should be
discontinued to avoid further resource expenditure. This stage does not
always indicate that the initiative is a failure, but rather that the
initiative is not aligned with the overall business strategy or is not
able to deliver value. This means that the implementation of the
initiative is not feasible or might not be needed anymore. In the best
case abonded initiatives are the ones which are able to provide
learnings and insights which can be used in future initiatives.

In order to properly decide whether to move to the next stage, the
correct questions need to be asked and answered. The following segment
will outline the key decision points for each of the stages:

\begin{longtable}{@{}
		>{\raggedright\arraybackslash}p{0.17\linewidth}
		>{\raggedright\arraybackslash}p{0.27\linewidth}
		>{\raggedright\arraybackslash}p{0.10\linewidth}
		>{\raggedright\arraybackslash}p{0.16\linewidth}
		>{\raggedright\arraybackslash}p{0.30\linewidth}
		@{}}
	\toprule
	\textbf{Stage} &
	\textbf{Key Decision Checkpoint} &
	\textbf{Possible Outcomes} &
	\textbf{Governance / Roles} &
	\textbf{Notes / Edge Cases} \\
	\midrule
	\endhead
	\bottomrule
	\endlastfoot
	
	\textbf{1. Opportunity Scouting} &
	Is the use case strategically relevant and feasible? &
	Proceed
	
	Iterate
	
	Reject &
	AI Sponsor
	
	Business Analyst
	
	EA Lead &
	Multiple overlapping opportunities → consolidate portfolio
	
	Unclear ROI → iterate \\
	\midrule
	
	\textbf{2. Strategic Alignment} &
	Does the initiative align with corporate strategy and transformation goals? &
	Proceed
	
	Iterate
	
	Reject &
	Steering Committee
	
	CTO &
	Conflicting priorities → defer or adjust scope \\
	\midrule
	
	\textbf{3. Technical Feasibility \& Data Assessment} &
	Are risks manageable? Is the use case feasible? &
	Proceed
	
	Iterate
	
	Reject &
	Risk Officer
	
	IT Department
	
	Adoption Manager &
	Legacy systems or missing data → iterate
	
	Severe gaps → reject \\
	\midrule
	
	\textbf{4. Pilot Design} &
	Is the pilot implementation plan feasible and properly scoped? &
	Proceed
	
	Iterate
	
	Reject &
	Pilot Manager
	
	Team Leads
	
	Governance Board &
	Dependencies on other pilots → integrate
	
	Lack of resources → iterate \\
	\midrule
	
	\textbf{5. Pilot Execution} &
	Does the pilot meet adoption and performance criteria? &
	Proceed
	
	Iterate
	
	Reject &
	Dev Team
	
	Data Scientists
	
	Adoption Managers &
	Low adoption → iterate with change management
	
	Technical failure → iterate or reject \\
	\midrule
	
	\textbf{6. Evaluation \& Scaling Decision} &
	Should the initiative be scaled across the organization? &
	Proceed
	
	Iterate
	
	Reject &
	Steering Committee
	
	Governance Board &
	Partial success → phased scaling
	
	Multiple pilots succeed → portfolio prioritization \\
	\midrule
	
	\textbf{7. Scale Deployment} &
	Is full deployment technically and operationally feasible? &
	Proceed
	
	Iterate
	
	Reject &
	Adoption Manager
	
	IT Ops
	
	Department Leads &
	Bottlenecks or workflow conflicts → iterate
	
	Systemic risks → reject \\
	\midrule
	
	\textbf{8. AI-Native Optimization} &
	Is continuous improvement and adoption sustainable? &
	Proceed
	
	Iterate
	
	Reject &
	AI Governance Board
	
	Process Owners &
	New use cases emerge → loop back to Step 1
	
	Tech upgrades → iterate \\
	
\end{longtable}


Each of the questions which are outlined per step were consciously
chosen in order to target the underlying challenges which are inherent
to each of the stages. The questions per stage try to target the main
risk and challange connected with the stage and thus by having a clear
understanding of the challenges, the teams are able to mitigate the
risks and are able to make informed decisions. The possible outcomes of
each decision point are designed to provide flexibility and
adaptability, allowing organizations to respond to new insights or
challenges as they arise. By incorporating these decision checkpoints,
SAIL ensures that AI initiatives are continuously evaluated for their
strategic fit, technical feasibility, and operational readiness, thereby
enhancing the likelihood of successful adoption and sustained value
delivery. In the table each question and step has an inherent connection
to the relevant stakeholder and roles which are involved in the
decision-making process. By having clear roles and responsibilities, the
decision-making process is able to be streamlined and is able to reduce
the risk of conflicts and bottlenecks. This also allows clear
communication and transparency throughout the entire adoption process as
each stakeholder knows where do they stand in the decision and
accountability process. The table is also enhanced by a column of notes
and edge cases which are outlining the potential edge cases and
scenarios which might occur during the decision-making process. These
are serving as a potential and expected bottleneck which the company
might be facing during the adoption process. The edge cases such as
overlapping pilots, conflicting priorities, legacy systems or missing
data are very common in the adoption process and by being aware of these
potential bottlenecks, the teams are able to plan ahead and reduce the
risk of failure and increase the probability of successfully passing the
decision points. Each 8 step decision cycle presents a unique set of
challenges and opportunities, and by having clear decision points,
roles, and potential edge cases, the SAIL framework is able to provide a
structured and transparent approach to AI adoption in software
organizations. This means that after each cycle the ability to loop and
start from the beginning is possible and is even encouraged as the
company might gather more Use Cases, knowledge and identify new fleet of
opportunities which are able to be solved with AI. The dynamic decision
making framework allows the company to be agile and responsive to the
ever-changing landscape of AI technology and its applications in the
software domain while ensuring the process of governance and structured
approach is not lost. The strategic placement of the decision points is
crucial as they are placed in the most logical and impactful stages of
the adoption process. The decision points are placed after individual
stages where the most critical decisions need to be made, ensuring that
the adoption process is continuously evaluated and aligned with
organizational goals.

\section{Detailed Adoption
	Stages}\label{detailed-adoption-stages}

In the following subsections, each of the eight stages of the SAIL
framework will be outlined from the perspective of objectives, key
activities, roles involved, decision points, and expected outputs. This
detailed breakdown will provide a comprehensive understanding of the
necessesity of each stage and how they interconnect to form a cohesive
AI adoption lifecycle. By granularly defining each stage and breaking it
down into its core components and the 5 primary elements, the analysis
will have a cohesive and structured approach to understanding the
underpinning principles of each stage and how they contribute to the
overall success of the AI adoption journey. This detailed breakdown will
serve as a practical guide for organizations looking to implement the
SAIL framework, providing clarity on what to expect and how to navigate
each phase effectively.

\subsection{Step 1: Opportunity
	Scouting}\label{step-1-opportunity-scouting}

Opportunity scouting is the initial stage of the SAIL framework, where
the focus is on identifying and prioritizing high-impact AI
opportunities within the organization. This stage will be setting the
foundation of the AI revolution in the company. It is the stage where
the ideas of the future are being born and where the potential for value
creation is being identified. The objective of this stage is to identify
the initial set of painpoints which the company is dealing with at the
current day and age. The idea generation will be led in a bottom-up
manner, where the engineers and the teams are able to propose use cases
which they are facing in their day to day work. By giving the engineers
a say and voice the company is receiving direct feedback from the people
who are actually doing the work and are able to identify the painpoints
and bottlenecks which are hindering their productivity. The idea
generation will be complemented by a top-down approach, where the
management and the business stakeholders are able to propose use cases
which are aligned with the overall business strategy and goals. By
creating a hybrid approach where idea generation comes from engineers
and is evaluated by upper management overseeing the vision, mission and
roadmap of the company, the company is able to create a balanced
portfolio of AI initiatives which are able to deliver value in the
short, medium and long term. The idea generation will be complemented by
a structured approach of evaluating the ideas based on their potential
impact and effort required to implement them. The ideas will be
evaluated based on an effort vs impact matrix, where the ideas with the
highest impact and lowest effort will be prioritized for further
evaluation. This structured approach will ensure that the company is
able to focus its resources on the most promising opportunities and is
able to deliver value in a timely manner. The idea generation process
shall be led by the Adoption Manager and the Sponsor, who will be
responsible for collecting and evaluating the ideas and ensuring that
they are aligned with the overall business strategy and goals. The
preferred outcome of this stage will be a shortlist of candidate use
cases which are able to deliver value and are aligned with the overall
business strategy and goals. This process shall be done in an open
manner where engineers have the freedom to propose ideas and are not
limited by the current technological capabilities or the current state
of the organization. In order to ensure that the idea generation process
is as efficient as possible, the Adoption Manager and the Sponsor shall
prepare a set of categories and themes which are aligned with the
overall business strategy and goals. Categories such as (e.g., customer
support, internal efficiency, product development, workflow automation)
will be used to guide the idea generation process and give engineers in
the ideation round a spark and direction in which they see potential for
AI to be implemented. The idea generation process shall empower the
engineers and the teams to think outside the box and propose ideas which
are not only solving current painpoints but are also pushing the
boundaries of what is possible with AI.

\begin{itemize}
	\item
	\textbf{Objective:} Identify and prioritize high-impact AI
	opportunities.
	
	The objective of this stage is to identify and prioritize high-impact
	AI opportunities inside of the organisation which are solving a real
	painpoint and are able to deliver value in a timely manner. he focus
	on value in terms of either efficiency, cost reduction or revenue
	generation is crucial as the AI initiatives are not only a
	technological experiment but are also a business initiative which is
	able to deliver value to the organisation. The objective of this stage
	will be the process of strategically placing the engineers in a closed
	environment where the idea generation process will lead them
	throughout the sessions with the key focus on innovation and value
	delivery which helps the company in not only short term but also long
	term manner
	\item
	\textbf{Key Activities:} Environmental scanning, idea collection,
	initial value estimation.
	
	The key activities during the opportunity scouting stage will be
	focused on environmental scanning, idea collection and initial value
	estimation. The need to create a structured approach to idea
	collection is crucial as it helps engineers think in a direction where
	the ideas are aligned with the overall business strategy and goals.
	After the initial idea collection, the ideas will be evaluated based
	on their potential impact and effort required to implement them. The
	ideas will be evaluated based on an effort vs impact matrix, where the
	ideas with the highest impact and lowest effort will be prioritized
	for further evaluation. By having outlined the set of key activities
	the teams and adoption manager will be able to approach this stage
	with a greater easy and ensuring the people that the framework they
	are following is structure and will prevent chaos and confusion.
	\item
	\textbf{Roles:} Sponsor, Adoption Manager, Teams.
	
	The key roles involved in the opportunity scouting stage will be the
	Sponsor, Adoption Manager and the Teams. In this stage the sponsor and
	the adoption manager will be responsible for leading the idea
	generation process and ensuring that the ideas are aligned the
	company's business strategy and goals. The teams and especially the
	engineers will be responsible for proposing ideas and providing
	feedback on the ideas which are being proposed. The engineers are the
	pinpoint of this stage as all the ideas are coming from them and they
	are the ones who are able to identify the painpoints and bottlenecks
	which are hindering their productivity, the role of Sponsor and
	Adoption Manager is to only efficiently channel the ideas and ensure
	that they are aligned with the overall business strategy and goals.
	\item
	\textbf{Decision Points:} Is the use case strategically relevant and
	feasible?
	
	The decision point whether to proceed, iterate or abandon the
	opportunity scouting stage will be based on the question ``Is this use
	case strategically relevant and feasible?''. The question shall be
	asked for every single one of the potentially identified and outlined
	usecases as it determines winners from losers. This question however
	dos not focus on strategy nor business alignment, it is rather a
	feasibility check to the round of engineers, whether this usecase
	occurs regularly or it does not need AI as a solution.
	\item
	\textbf{Output:} Shortlist of candidate use cases.
	
	The expected output of this stage will be a set of potential use cases
	which are not analysed but rather a good starting point for the next
	stage of the adoption process. The shortlisted use cases are not the
	final use cases as they were not properly analysed, nor alligned with
	the overall business strategy and goals and were not risk assessed,
	but shall rather serve as a state of mind of the engineers dealing
	with pain points and bottlenecks in their day to day work.
\end{itemize}

\subsection{Step 2: Strategic Alignment
	Review}\label{step-2-strategic-alignment-review}

The process of strategic alignment review is the second stage of the
SAIL framework in which the focus comes on the list of use cases drafted
by the engineers. This provess envolves the governance board and the
sponsor who are able to evaluate the use cases based on their strategic
fit and alignment with the overall business strategy and goals. The
objective in this case will be the process of ensuring that the use
cases are aligned and are relevant for the business. Throughout this
stage each of the use cases will be evaluated by the responsible teams
and the initial analysis will be done. This analysis will be done by
both teams and the sponsor together with the governance board. This
analysis will be done in a dual format due to the need to find realistic
expections.

The initial analysis done shall follow this structure:

\begin{description}
	\item[What is your idea?] 
	\begin{description}
		\item[The Problem we are solving:] Describe the specific problem or pain point this project addresses.
		\item[The Solution:] Explain your proposed solution and how it works.
		\item[Who will use it?] Identify the target users and stakeholders.
	\end{description}
	
	\vspace{1ex}
	\item[Why is this valuable?] 
	\begin{description}
		\item[Business Benefits:] How does this help the business? What value does it create?
		\item[Time Savings:] How much time will this save? Per day/week/month?
		\item[Daily Work Impact:] How will this make people's daily work better/easier/faster?
	\end{description}
	
	\vspace{1ex}
	\item[How will you build it?] 
	\begin{description}
		\item[Implementation Plan:] 
		\item[What do you need?] What tools, systems, or resources are required?
		\item[How long will it take?] Rough estimate for development and implementation.
		\item[What could go wrong?] Main risks or challenges you're worried about.
	\end{description}
	
	\vspace{1ex}
	\item[How will you know it's successful?] 
	\begin{description}
		\item[Success Measures:] List specific, measurable criteria that will indicate success.
	\end{description}
	
	\vspace{1ex}
	\item[Current Status \& Next Steps]
	\begin{description}
		\item[Where are we heading?] Describe current status and immediate next steps.
	\end{description}
\end{description}


By being able to answer this set of questions, the teams and the
governance board are able to have a dicussion and are able to evaluate
the use cases based on their strategic fit and alignment with the
overall business strategy and goals. This analysis gives the teams a
sense of ownership and responsibility which improves the quality of the
use cases and gives a new perspective on the use case to the governance
board and the sponsor. The important questions such as ``Why is this
valuable?'' and ``How will you know it's successful?'' are crucial for
the success of the adoption process as they help the teams and the
governance board to focus on value delivery and measurable outcomes. The
questions of ``Why is this valuable?'' and the 3 subquestions are
crucial as they help the teams and the governance board to focus on
value delivery and measurable outcomes. The key of them is not a precise
estimation but rather a guidance for the governance board to understand
the impact of the use case and the ability to capitalize from successful
execution.

\begin{itemize}
	\item
	\textbf{Objective:} Ensure selected opportunities align with business
	strategy.
	
	The key objective of this stage is to ensure that the use cases which
	were drafted in the previous stage are aligned with the overall
	business strategy and goals. This being achieved by a dual analysis of
	the use cases by both the teams and the governance board. The
	objective of this stage will be to ensure that the drafted use cases
	are prioritized by the governance board and the sponsor based on their
	strategic fit will be considered when the teams are walking down the
	line of the SAIL framework.
	\item
	\textbf{Key Activities:} Stakeholder review, alignment workshops.
	
	The key activities throughout the strategic alignment review stage
	will be stakeholder review and alignment workshops, which represent
	the indepth use cases analysis. The activity of analysis the use case
	picked by the team creates ownership which is desired by engineers as
	the feeling of belonging and responsibility serves as driving force
	for the success of the adoption process. The alignment workshops will
	be led by the sponsor and the governance board, will serve as a
	discusson with review, essentially leading to cherry picking the set
	of use cases which are able to either proceed to the next stage, need
	to be iterated on or need to be abandoned.
	\item
	\textbf{Roles:} Governance Board, Sponsor, Teams.
	
	The roles in the strategic alignment review stage are the Governance
	Board, Sponsor and the Teams. As already mentioned previously the
	Governance Board and the Sponsor are the ones working together to
	creating one side of the analysis coming from the perspective of the
	business and strategy while the teams are the ones providing the
	technical perspective and the feasibility of the use case. The dual
	analysis concept comes to the closing stages once the final discussion
	is held and the decision of whether to proceed, iterate or abandon the
	use case is made. The ability to yield both perspectives is crucial as
	it helps the teams and the governance board to have a holistic view of
	the use case and its potential impact on the organization.
	\item
	\textbf{Decision Points:} Does the initiative align with corporate
	strategy and transformation goals?
	
	The decision point whether to proceed, iterate or abandon the
	strategic alignment review stage will be based on the question ``Does
	this initiative align with corporate strategy and transformation
	goals?''. The question needs to be asked with the mind of not only the
	business value and strategy but also the technical feasibility and the
	actual impact which the engineers estimated. After the analysis is
	done, the concrete numbers of efficiency gains, time savings and daily
	work impact will be used to evaluate the use case and the impact
	leaving on the organisation and its internal processes
	\item
	\textbf{Output:} Approved use cases for assessment.
	
	The output of this stage will be a set of approved use cases which are
	able to proceed to the next stage of the adoption process. This will
	be a set of use cases which are critical for the business, create
	value, improve either efficiency or reduce costs and are aligned with
	the overall business strategy and goals. This approved set of use
	cases will be the backbone of the adoption process and will be the use
	cases which will be evaluated in the upcoming stage of feasibility and
	risk assessment. By having a great understanding of the usecases and a
	deep down analysis done by both business side and technical side the
	probability of success is greatly increased and the use cases are able
	to be measured by concrete goals.
\end{itemize}

\subsection{Step 3: Feasibility \& Risk
	Assessment}\label{step-3-feasibility-risk-assessment}

The next stage in the framework is the stage of feasibility and risk
assessment, where the focus comes on the technical feasibility and
thorough risk assessment process. This stage is the first stage where
the technical perspective is coming into play and is the stage where
dreams meet reality and might often turn into dust as the idea is faced
with the current IT challanges and obstackles which have to be either
resolved or the idea will have to be disolved and lesson will have to be
learned that similiar scope or expectation might not be feasible unless
the IT infrastructure changes. This could either be (eg. license issues,
data availability, technical complexity, migration dependecies). From
the perspective of the risk assessment, the focus will be on identifying
and mitigating potential risks associated with the use case. The risk
assessment will be done in a structured manner, where the risks will be
identified, evaluated and based on the severity and potential obstacles
resolved appropriately. Risk assessment as a concept will have to be
done by a proper Risk Officer as they are able to provide an in-depth
knowledge on the domain and have the highest chance of catching the
errors early on and preventing catastrofic failures down the line. The
risk assessment has a stigma around it for the engineers which see it as
a bureaucratic process declining innovation and creativity, but in
reality the risk assessment is a crucial process which helps the teams
and the organisation to identify potential risks and mitigate them
before they become a problem. By having a serious risk assessment
process, the teams and the organisation are able to commit to work on
use cases which are ``Risk Assessed'' and are able to work on projects
which are not likely to be declined down the line due the risks which
were not identified early on.

\begin{itemize}
	\item
	\textbf{Objective:} Assess technical, organizational, and risk
	factors. The objective of this stage is to assess the technical,
	organizational and risk factors associated with the use case. The
	focus of this stage is the identification of potential bottlenecks and
	risks which are associated with the use case and the ability to
	mitigate them before they become a problem. From the technical
	perspective this stage is not only about declining due to lack of
	infrastructure but also about pointing which resources can be
	levaraged in order to fullfil the use case.
	\item
	\textbf{Key Activities:} Technical feasibility study, risk analysis,
	resource check.
\end{itemize}

The key activity during this stage will be the technical feasibility
study, risk analysis and resource check. From the perspective of
technical feasibility study the usage of the framework such as (e.g.,
Technology Readiness Level (TRL) assessment, Data Maturity Model or AI
Capability Maturity Model) could be used to evaluate the technical
feasibility of the use case. The risk analysis is rather more tricky as
each company is different and the field they are operating in requires
different level of risk assessment. The common risk analysis frameworks
such as (e.g., Failure Mode and Effects Analysis (FMEA), SWOT Analysis,
or Risk Matrix) could be used to evaluate the risks associated. By doing
so the key risks are able to be identified and further actions shall be
taken in order to alleviate them

\begin{itemize}
	\tightlist
	\item
	\textbf{Roles:} Risk Officer, IT Department, Adoption Manager
\end{itemize}

The roles of this stage will be the Risk Officer, IT Department and the
Adoption Manager. The main driver of the feasibility will be the
combination of the IT Department and the Adoption Manager as they will
work together in order to evaluate whether the use case is technically
feasible and whether the resources are available. The teams are not
involved in the process as the Adoption Manager is representing them and
ensuring that their perspective is taken into account. The perspective
of the IT Department is advocated to ensure that the strategic resources
are available and that the use case is technically feasible. The Risk
Officer is the one who is able to provide the risk assessment and ensure
that the risks are identified and mitigated before they become a problem
as they are the expert on the topic and their opinion is the most
relevant on the topic.

\begin{itemize}
	\tightlist
	\item
	\textbf{Decision Points:} Are risks manageable? Is the use case
	feasible?
\end{itemize}

In this stage there are in fact 2 questions which need to be answered
and they both must be answered with a ``Yes'' in order to proceed. The
questions being: ``Are the risks manageable?'' and ``Is the use case
feasible?''. The first question focusing whether the impact of the risk
and the followed mitigation actions are worth the effort and the
resources which are needed to be allocated. The second question focusing
on the technical feasibility and whether the actual use case and its
implementation is feasible in the current state of the IT infrastructure
and its disposable resources. The need for 2 questions is crucial as
this decision points acts on 2 fronts, the technical feasibility one and
the risk assessment one. Both of them are very crucial and none of them
shall be neglected.

\begin{itemize}
	\tightlist
	\item
	\textbf{Output:} Go/no-go decision for pilot. The output of this stage
	closes the analytical and preparatory part of the SAIL framework where
	we have a justified idea from every single point, but the stage 3 is
	focused only on delivering on the questions of if the company's
	digital and physical assets are ready for a new use case and if the
	risks associated with its implemetation are a threat to the company.
	The output of this stage will be a go/no-go decision for the pilot
	stage. Once the decision to proceed is made, the use case is able to
	proceed to the next stage of the adoption process, being the pilot
	design stage. If the outcome of this stage will be to iterate the
	process will not be light as once the reason to iterate is a risk
	based one, the idea might be in jeopardy and the ability to iterate
	will be limited. If the outcome of this stage is to abandon the use
	case, the use case will be documented and lessons learned will be
	gathered in order to improve the process in the future.
\end{itemize}

\subsection{Step 4: Pilot Design}\label{step-4-pilot-design}

The pilot design stage is the stage in which the analysis is being
transformed into action points and stories. In order to transform an use
case into pilot the appropriate steps needs to be defined. The steps of
each individual pilot in terms of its implementation plan are different
and that is for the Adoption Manager and the Pilot Team to define. The
pilot design stage is the stage where the use case is being broken down
into smaller chunks and is being transformed into actionable items. The
pilot design stage is crucial as it sets the foundation for the pilot
execution stage and ensures that the pilot is properly scoped, resourced
and governed. The steps from initial setup of repositories to
architecture of the custom tool (if needed) is created and the tickets
are spawned and assigned to respective team memebers. This is also the
the stage in which the team starts to build confidence as the use case
matures and starts to be something which the teams are able to see
outlined infront of them like an actual project. The pilot design stage
is the stage where the teams are able to see the light at the end of the
tunnel and are able to see the potential of the use case. While they are
still not implementing and solving the use cases proper planning and
roadmap creation prepares the team for the oncoming future

\begin{itemize}
	\tightlist
	\item
	\textbf{Objective:} Plan the pilot implementation. The objetive of
	this step is the creation of a roadmap filled with milestones and
	actionable itmes which serve as tickets for the engineers to implement
	down the road. The goal here is to outline precise steps and direction
	for the team to take in order to convert the use case to a pilot
	project. The heavy lifting of this stage is done by the Adoption
	Manager who is responsible for the roadmap creation and the pilot team
	who is responsible for the technical input in regards to the
	implementation plan. The goal of this stage is to create a clear and
	concise plan which is able to be followed by the team and is able to
	deliver value in a timely manner.
	\item
	\textbf{Key Activities:} Roadmap creation, governance setup,
	refinement of success criteria.
\end{itemize}

The key activities which are done throughout this stage is the roadmap
creation, governance setup and refinement of success criteria. The main
activity is the roadmap creation which serves as the backbone of the
project and its importance does not only lie in the team having a clear
direction but also in the ability to communicate the plan to the
stakeholders and the governance board. This actionable backlog gives the
management trust and ability to track the work of the engineers on a
cross-team basis. Very important activity done during this stage is the
refinement of the roadmap with engineers. Once a proper discussion on
the roadmap starts the engineers are able to share their feedback not
only on the strategy but also the complexity, thus helping on the time
and effort estimation.

\begin{itemize}
	\item
	\textbf{Roles:} Adoption Manager, Pilot Team. The involved
	stakeholders in this stage are the Adoption Manager and the Pilot
	Team. The Adoption Manager is the one who is responsible for the
	creation of the roadmap and the pilot team is the one who is
	responsible for providing the appropriate technical input and ensuring
	that the roadmap is feasible and is able to be implemented in a timely
	manner. The Adoption Manager leads this stage from their front and
	works closely as a team with the Pilot Team to enhance the outcome.
	\item
	\textbf{Decision Points:} ``Is the pilot implementation plan feasible
	and properly scoped?''
\end{itemize}

This stage focuses on creation of the pilot plan and its feasibility and
therefore the most crucial question which needs to be answered is ``Is
the pilot implementation plan feasible and properly scoped?''. The
question was chosen in order to emphasize the feasibility of the use
cases and now the actual pilot design. The scope is another variable
which needs to be addressed and critically evaluated as the scope creep
is a common problem in the pilot execution stage and by having a clear
scope and boundaries the team is able to focus on delivering value and
not on expanding the scope. Working on tasks and features outside of the
scope hinders the value-delivery time even if the features are nice to
have.

\begin{itemize}
	\tightlist
	\item
	\textbf{Output:} Pilot implementation plan.
\end{itemize}

The clear output of this Pilot Design stage is a well defined pilot
implementation plan which is able to be followed by the team and a clear
scope which is able to be communicated to the stakeholders and the
governance board. The plan for the team creates a vision for the team
and ensures the team that there is a clear path outlined in front of
them and on the end of the path is a successful pilot which is able to
solve their daily painpoints and bottlenecks. The plan is not set in
stone and is able to be iterated upon, however by having a good and well
refined plan the probability of failure is greatly reduced.

\subsection{Step 5: Pilot
	Execution}\label{step-5-pilot-execution}

Pilot Execution is the stage in which the actual implementation of the
pilot takes place. This stage involves gradual ticket pulling from the
backlog alligned with the roadmap in order to build the pilot solution.
The pilot execution stage is the stage where the teams are building the
pilot ticket after ticket while the Adoption Manager is responsible for
ensuring that the pilot is properly governed and that the success
criteria are being met. The pilot execution stage can be structured
according to the teams usual project management structure such as (e.g.,
Agile, Scrum, Kanban) in order to ensure that the team is used to the
work dynamic and do not have to go through a change management process.
This stage ends once the first MVP is delivered and it can be tested by
users in order to give feedback and validate the solution. This however
is done in a controlled environment and the pilot team works in a cyclic
manner in order to ensure that the solution is properly tested and
validated before it is rolled out to a larger audience and the feedback
is always considered.

\begin{itemize}
	\tightlist
	\item
	\textbf{Objective:} Build and test the AI solution in a controlled
	environment.
\end{itemize}

The objective of this stage is to build and test the pilot project and
verify its functionality with a first set of users. The focus of this
stage is to deliver a working solution which is able to be analysed and
an impact of it can be measured. This impact or otherwise known as
success criteria is defined in the analysis stage and taken into
consideration during the pilot execution stage. Once the pilot does not
meet the success criteria the team is able to iterate and improve the
solution until the success criteria are met.

\begin{itemize}
	\tightlist
	\item
	\textbf{Key Activities:} Development, testing, user feedback
	collection.
\end{itemize}

The leading activities performed during this stage are development,
testing and user feedback collection. This stage starts with the
development of the pilot and only once a working solution is implemented
the first batch of users shall provide feedback and observances in the
shortest amount of time in order for the team to refine their solution.
The thorough testing and feedback collection helps mitigate initial
problems and ensures that the solution is impactful and user-friendly.

\begin{itemize}
	\tightlist
	\item
	\textbf{Roles:} Pilot Team, End Users.
\end{itemize}

During this stage there are 2 main roles involved, being the Pilot Team
and the End Users. The Pilot team, being the one responsible for the
actual implementation of the pilot and the end users, being the ones who
are able to provide feedback and validate the solution. The end users
are crucial for the success as their feedback helps improve the product
as their opinions do not have internal bias and are new to the solution.

\begin{itemize}
	\tightlist
	\item
	\textbf{Decision Points:} Does the pilot meet adoption and performance
	criteria?
\end{itemize}

The question to be asked whether this stage is able to proceed or not is
``DoesDoes the pilot meet adoption and performance criteria?''. At the
end of the pilot execution stage the team is able to evaluate whether
the success criteria which were defined in the analysis stage are met or
not. If the success criteria are met the team is able to proceed to the
next stage of evaluation and scaling decision. The benefit of defining
the criteria prior to the exeution stage is that emotions can be
isolated as the decision is purely based on the merit.

\begin{itemize}
	\tightlist
	\item
	\textbf{Output:} Pilot results and lessons learned.
\end{itemize}

The output of this stage will be the pilot results and lessons learned.
This means that there will be a working MVP of the use case outlined at
the beginning which serves as valude delivery instrument having a
potential to directly influence the enterprise and create feasible
impact. The great benefit of having a structured approach is that
throughout each of the decision on the roadmap is that there is a set of
lessons learned on the way which can be used in the next iteration of
the framework.

\subsection{Step 6: Evaluation \& Scaling
	Decision}\label{step-6-evaluation-scaling-decision}

The next stage in the SAIL framework is the stage of evaluation and
scaling decision, the focus is now suddenly placed on the evaluation of
the pilot results and the decision whether to continue the initiatives
implemented in the previous stage of implementation. This decision shall
not be taken lightly as this stage is a make it or break it. The
decision to scale the pilot to a larger audience or to abandon the
initiative is crucial as it determines whether more resources such as
finances and time have to be invested into scaling and infrastructure
purchasing. This decision shall be based on the evaluation of the pilot
results coupled with the analysis done in the step 2 and 3. The decision
to scale is solely focused on ROI and potential usability of the pilot
solution. The evaluation process shall be done in a structured manner,
by the governance board and the sponsor, as their opinion on the topic
comes from the high level overview and are the most relevant
stakeholders to make the decision.

\begin{itemize}
	\item
	\textbf{Objective:} Evaluate pilot outcomes and decide on scaling. The
	objective of this stage will be to come to a decision. The decisions
	in this stage which we have are 4. Give a green light to scale and
	deploy, to iterate and improve the initial pilot to optimize it for
	better rollout, to abandon the initiative as it does not create value
	or to pivot and change the direction of the initiative and to leave it
	as it is. The last decision is something the engineers do not like as
	it is a sign of failure, but in reality it is a sign of maturity and
	understanding that not all ideas are good and not all ideas are
	feasible to be scaled. If the pilot stays in the pilot stage and is
	able to already deliver value and its scaling is not feasible, the
	decision to leave it as it is might be the best one.
	\item
	\textbf{Key Activities:} Analyze results, stakeholder review,
	cost-benefit analysis. The main activities done during this stage are
	the analysis of the results, stakeholder review and cost-benefit
	analysis. By having a great overview on the scaling and deployment
	costs the governance board and the sponsor are able to make an
	informed decision whether to scale or not. The ability to
	differentiate between the cost of scaling and the benefit of scaling
	is crucial as by doing so the calculation of the ROI is possible and
	data driven strategic decision can be done with high confidence.
	\item
	\textbf{Roles:} Governance Board, Sponsor, Adoption Manager.
	Responsible entities for this stage are the Governance Board, Sponsor
	and the Adoption Manager. Cost analysis and benefit analysis is done
	by the Adoption Manager while Sponsor is left to make the final
	decision together with the Governance Board. Their theoretical
	knowledge and high level overview comes very handy in this stage as
	they are able to see the bigger picture which often overhauls the
	technical perspective.
	\item
	\textbf{Decision Points:} Is scaling justified based on ROI and
	impact? The decision point whether to scale or not is based on the
	question ``Is scaling justified based on ROI and impact?''. This
	question directly underpins the core priniciple of this entire stage.
	Whether to scale or not is a very difficult decision as sometimes the
	impact is great but the cost of scaling is too high and sometimes the
	cost of scaling is low but the impact is negligible. The second
	alternative of simply leaving the pilot as it is and not scaling it is
	often overlooked, however it is a valid decision as sometimes the best
	one.
	\item
	\textbf{Output:} Scale/no-scale decision. Simplest summation of this
	stage is the output, being the scale/no-scale decision. This outcome
	can be the decision to scale or not to scale. This decision either
	moves us down the line of the SAIL framework or it stops the process
	and the use case needs to be documented and lessons learned need to be
	gathered, but the pilot will be either depricated or will stay as it
	is without further scaling. By going to this dead end the company goes
	back to the drawing board and starts the process from the beginning,
	however this time with more knowledge and experience, which helps the
	company to avoid going down the same path and making the same
	mistakes.
\end{itemize}

\subsection{Step 7: Scale
	Deployment}\label{step-7-scale-deployment}

One of the most exiting and impactful stages for the entire company is
the stage of scale deployment. This stage transform internal pilots into
full AI solutions which are able to be used by the entire organization.
The scale deployment stage is the stage where the focus comes on rolling
out the AI solution to a larger audience and collecting immediate
feedback. In order to ensure successful scaling the AI solutions needs
to be thoroughly tested and validated in the pilot stage, as the first
impression is the most important one. People tend to base their trusts
on the first impression and if the solution is sub-par, not sufficient
and users do not see immediate value, the adoption will be low and the
solution will be abandoned and not supported by the end users. The scale
deployment stage is both technical but also organizational as the change
management process is crucial for the success of the scaling. The
creation of awareness is extremely important as it creates a buzz around
the solution and helps the end users to understand the value of the
solution.

\begin{itemize}
	\item
	\textbf{Objective:} Roll out the solution organization-wide.
	Transformation from pilot to AI solution. This action is the sole
	objective of this stage. Transforming the pilot into wide adopted AI
	solution comes with a lot of challenges, which are overcome by proper
	planning and execution. The objective is to have a smooth rollout and
	infrastructural resources to support the solution. Successful scaling
	is not only about the technical implementation but also about the user
	stories which the solution is able to solve.
	\item
	\textbf{Key Activities:} Scaling, infrastructure setup, change
	management.
\end{itemize}

Key activities done during this stage are scaling, infrastructure setup
and change management. The most important activity for successful
scaling is the technological readiness. The infrastructure needs to be
able to handle concurrent users while not experiencing downtime or
latency. The infrastructure needs to be scalable and flexible in order
to adapt to the changing needs of the organization. This simple activity
can be solved by using a single variable - money. The more money is
invested into the infrastructure the more reliable and scalable it
becomes.

\begin{itemize}
	\tightlist
	\item
	\textbf{Roles:} Adoption Manager, IT, Business Units.
\end{itemize}

The roles involved in this stage are the Adoption Manager, IT and
Business Units. The IT being responsible for technical scalability and
infrastructural readiness, the Adoption Manager being responsible for
the change management process and the Business Units being responsible
for the user adoption and feedback collection, while working closely
with the Adoption Manager. Their close collaboration is crucial as it
will have impact on the user adoption as they will be promoting the
solution and creating awareness around it. The Business Units are the
ones who are able to provide direct feedback on the solution due to the
fact that in each business unit there are different painpoints and use
cases which the solution might solve.

\begin{itemize}
	\item
	\textbf{Decision Points:} Is the organization ready for full-scale
	deployment? The decision point whether to proceed with the scaling or
	not is based on the question ``Is the organization ready for
	full-scale deployment?''. This question is crucial as it determines
	whether the organization is ready for the change and whether the end
	users are ready to adopt the solution. The readiness of the
	organization is determined by the change management process and the
	user adoption. If the organization is not ready for the change, the
	scaling will fail and the solution will be abandoned. The readiness of
	the organization is determined by the change management process and
	the user adoption. If the organization is not ready for the change,
	the scaling will fail and the solution will be abandoned.
	\item
	\textbf{Output:} Scaled AI solution. The genaral outcome of this stage
	is a scaled AI solution which delivers sustainable value over time.
	The scaled AI solution is able to solve the painpoints and bottlenecks
	of the end users and is able to either cut time to delivery, reduce
	costs or improve efficiency. The scaled AI solution shall be verified
	and validated by end users and the feedback shall be collected by the
	Business Units in order to forward the findings to the Adoption
	Manager who is able to iterate and improve the solution over time with
	the engineering team. The scaled AI solution is the backbone of the AI
	adoption process and is the foundation for the next stage of AI native
	optimization.
\end{itemize}

\subsection{Step 8: AI Native
	Optimization}\label{step-8-ai-native-optimization}

The final stage in the SAIL framework is the stage of AI native
optimization. The last stage of this lifecycle is the stage where the
focus comes on continuous improvement and optimization of the AI
solution happens. During this stage the AI solution is already
implemented and delivering value, however the probability of it being
perfect and solving all the painpoints is very low. The AI native
optimization stage is a continous process lasting over time, where the
focus comes on continuous improvement and optimization of the AI
solution. This stage however is problematic due to the fact on what to
decide as ``cut off date'' for the optimization. The optimization
process is a never ending story as there is always room for improvement
and optimization. What shall not be forgotten is the rule of diminishing
returns, where the more time and resources are invested into the
optimization the less value is delivered. The optimization process shall
be done in a structured manner and shall be based on the feedback
collected from the end users up until the moment the amount of request
slows down over time.

\begin{itemize}
	\tightlist
	\item
	\textbf{Objective:} Integrate, optimize, and continuously improve the
	AI solution.
\end{itemize}

The final goal of the AI native optimization stage is to integrate,
optimize and continuously improve the AI solution. This means that focus
is placed on the continuous improvement of the solution and its
maintenance. The optimization process of incrementally adding new
features and improving the existing ones is beneficial however the focus
shall be placed on the maintenance rather already innovative features.
This will ensure that the solution stays in the scope and will be used
by the end users for the intended purpose. If too many requests are
coming in, the solution might drift away from its original purpose and
the end users will not be able to use it for the intended purpose. In
such a case a new AI use case shall be created and the process starts
from the beginning, however for a different use case.

\begin{itemize}
	\tightlist
	\item
	\textbf{Key Activities:} Performance monitoring, process
	reengineering, feedback loops.
\end{itemize}

Activities done during this stage are performance monitoring, process
reengineering and feedback loops. The majority of work done during this
stage is performance monitoring and feedback loops. The AI solution
needs to be monitored in order to ensure that it is delivering value and
that end users are satisfied not only with the value delivery but also
with the user experience. The feedback loops are crucial to ensure high
user engagment and adoption. This process shall be done with using
interviews, surveys and in solution feedback collection forms. These
activities ensure smooth operation and high user satisfaction.

\begin{itemize}
	\tightlist
	\item
	\textbf{Roles:} Adoption Manager, Engineering Team, End Users.
\end{itemize}

The last stage of the SAIL framework involves the Adoption Manager,
Engineering Team and End Users. The key driver in this stage are the End
Users as they are the ones who are able to provide direct feedback on
the solution and are the ones not only using the tool but are also the
ones providing the most valuable comodity in the process, being the
feedback. The feedback is collected by the Adoption Manager and is
forwarded to the Engineering Team who is responsible for the
implementation of the feedback and the continuous improvement of the
solution. The Adoption Manager is the one who is responsible for the
overall process and ensuring that the feedback is collected and
forwarded to the Engineering Team in a timely manner.

\begin{itemize}
	\item
	\textbf{Decision Points:} Is the solution delivering sustained value?
	Are further optimizations needed? Whether the AI solution initiative
	can be marked as successful and the SAIL framework can be closed or
	not is based on the question ``Is the solution delivering sustained
	value? Are further optimizations needed?''. These questions are
	crucial for the final decision as the question of sustained value
	delivery is the main reason why the initiative started and can be
	evaluated by the end users. The second question of whether further
	optimizations are needed is crucial as it determines whether the
	solution is able to be improved or not. This can be decided by
	calculation of a moving average of the feedback and the number of
	requests coming in per unit of time. If the number of requests coming
	in over time is on a downward trend the team can plan to sunset the
	solution and start working on a new use case.
	\item
	\textbf{Output:} AI-native business processes.
\end{itemize}

The final outcome of the entire SAIL framework is the AI-native business
processes. This means that the company is able to yield sustainable
value over time and is able to continuously improve and optimize the AI
solution. The company have officially cleaned a painpoint which was
outlined in the initial stage and was successfully solved by leveraging
AI. This outcome marks as the ultimate goal of employment AI in the
enterprise and is the foundation for future AI initiatives. The company
does not only solve the pain point, but also gains extremely valuable
experience and knowledge which is able to be leveraged in the future,
not only for internal use cases but also for external use cases and
product development. The usage and integration of AI in the enterprise
is a never ending story and the SAIL framework is the foundation for
future AI initiatives and the backbone for the AI adoption process.

\section{Summary of the
	Framework}\label{summary-of-the-framework}

The SAIL framework provides a structured, lifecycle-based approach to AI
adoption in software organizations. It integrates decision checkpoints,
governance, and feedback loops to ensure scalable, coordinated, and
reusable AI initiatives. This comprehensive table summarizes the key aspects of the SAIL framework and extracts them in one place for easier reference.

\begin{table}[H]
\centering
\footnotesize
\begin{tabular}{|p{2.0cm}|p{2.3cm}|p{2.8cm}|p{2.3cm}|p{2.3cm}|}
\hline
\textbf{Step} & \textbf{Objective} & \textbf{Key Activities} & \textbf{Outputs} & \textbf{Governance Checks} \\
\hline
1. Opportunity Scouting & Identify and prioritize AI use cases. & Problem discovery, stakeholder engagement, feasibility analysis. & Prioritized use case backlog. & Is the use case strategically relevant? \\
\hline
2. Strategic Alignment Review & Align opportunities with strategy. & Dual analysis by teams and governance board. & Approved use cases. & Does it align with strategy? \\
\hline
3. Feasibility \& Risk Assessment & Assess technical and risk factors. & Feasibility study, risk analysis, resource check. & Go/no-go decision. & Are risks manageable and use case feasible? \\
\hline
4. Pilot Design & Define pilot implementation steps. & Roadmap creation, governance setup, success criteria. & Pilot implementation plan. & Is the plan feasible and scoped? \\
\hline
5. Pilot Execution & Build and test AI solution. & Development, testing, user feedback. & Pilot results and lessons learned. & Does pilot meet performance criteria? \\
\hline
6. Evaluation \& Scaling Decision & Evaluate pilot and decide on scaling. & Results analysis, stakeholder review, cost-benefit analysis. & Scale/no-scale decision. & Should initiative be scaled? \\
\hline
7. Scale Deployment & Roll out solution organization-wide. & Change management, system integration, MLOps setup. & Deployed AI solution at scale. & Is deployment feasible? \\
\hline
8. AI-Native Optimization & Continuously improve AI solution. & Performance monitoring, process reengineering, feedback loops. & AI-native business processes. & Is continuous improvement sustainable? \\
\hline
\end{tabular}
\caption{SAIL Adoption Stages}
\label{tab:sail_adoption_stages}
\end{table}

Table~\ref{tab:sail_adoption_stages} is clearly showcasing all of the eight steps which are required to successfully complete the SAIL framework. This table is consisting out 5 scopes. The first being of the name of the step in the SAIL framework, followed by the objective which the step is aiming to do. The next column is focusing on the key activities which are required to be done in order to successfully complete the step. The second to last column focuses on the outputs which should be expected by the end of the step and the last column is focusing on the internal governance check.  

By having a clearly outlined table, it is easier to reference, analyze and view how the entire SAIL framework operates and what are the key components of each individual step. This structured pathway is ensuring that AI initiatives are going to be strategically aligned, technically validated and verified in terms of the organizational governance check. 

\chapter{Use Case Implementation \&
	Validation}\label{use-case-implementation-validation}

\section{Introduction}\label{introduction-1}

This chapter will present an empirical application of the SAIL framework
in realistic software engineering contexts. By walking through how SAIL
is able to be applied to a diverse set of 4 use cases, the practical
utility and adaptability of the framework will be demonstrated. Each use
case will illustrate different entry points, challenges, and outcomes,
showcasing SAIL's flexibility across various organizational scenarios.

\section{Selection of Use Cases}\label{selection-of-use-cases}

The following use cases were carefully selected in order to represent
the diversity of AI adoption scenarios in software organizations. They
cover a range of functions from developer productivity to internal
knowledge management, demonstrating SAIL's applicability across
different domains and organizational needs. The use cases also vary in
complexity, data requirements, and stakeholder involvement, providing a
comprehensive validation of the framework.

\textbf{Selected use cases:}
\begin{longtable}{@{}
		>{\raggedright\arraybackslash}p{0.36\linewidth}
		>{\raggedright\arraybackslash}p{0.64\linewidth}
		@{}}
	\toprule
	\textbf{Use Case} & \textbf{Description} \\
	\midrule
	\endhead
	\bottomrule
	\endlastfoot
	
	AI-based Code Documentation Generation &
	Automated the process of generating and maintaining code documentation using AI models to analyze codebases and produce up-to-date documentation. \\
	
	AI-driven refactoring & 
	Introduced a new way how engineers tackle tasks which are related to optimising refactoring process by AI \\
	
	Internal LLM Chatbot for Knowledge Sharing &
	Deploying an internal large language model (LLM) chatbot to facilitate knowledge sharing and quick information retrieval among employees. \\
	
	Pull Request Review Automation &
	Automated the review process of pull requests using AI to identify potential issues, suggest improvements, and ensure code quality standards are met. \\
	
\end{longtable}


\section{Application of SAIL to Use Cases}\label{application-of-sail-to-use-cases}

Each use case is mapped through the SAIL lifecycle, highlighting entry
points, progression through stages, key decision points, governance
involvement, and outputs. The following subsections provide concise
walkthroughs for each use case.

\subsection{AI-based Code Documentation Generation}\label{ai-based-code-documentation-generation}

\textbf{Entry Point:} The first use case in which we are going to explore the implementation process in the AI testing framework is AI-based code documentation generation.

This use case will be focused on the most tedious and boring task for any software engineer: the creation and ongoing maintenance of code documentation. The process of writing code documentation is very often overlooked while being too focused on the actual code writing. Once any engineer gets the motivation to write the documentation, the documentation is often not at the quality and understanding level required for other engineers to understand the code. This insufficient documentation is then leading to a chain of problems such as misunderstanding of the code, lowered efficiency and sometimes even bugs in the code due to the fact that the engineers do not understand the code properly. The issue is clearly present in most of the software organizations and by having this problem solved the efficiency of the engineers will increase drastically.

In order to verify the SAIL framework we will explore how this use case was implemented in a realistic software engineering context.

\textbf{Opportunity Scouting:} 

This use case was identified by software engineers at the time and was initiated by the frustration of doing the task and not having the documentation properly up to date. The actual process of writing the documentation was not the hardest problem which the engineers have to tackle but the process of constantly changing it once the code changes.


\textbf{Strategic Alignment:} 

The process of strategic alignment for this specific use case was done in the context of the overall strategic goals of the company. The first step was the consultation with the management team in order to understand the strategic goals of the company and how this use case is able to align with them. The internal company goals were focused on the improvement of the company's efficiency and thus this use case was able to get the green light. The second step that needed to be done was the selection procedure for the appropriate engineers to implement this use case. A set of 3 engineers were selected to tackle it. For the simplicity of the entire process these engineers were from one single team and thus dealing with their line manager became easier and more streamlined. By solving both sides of the equation, the strategy as a company and strategy as a process this use case is now able to proceed to the next stage.

In terms of the expected metrics the team have outlined the following ones:
\begin{itemize}
	\tightlist
	\item
	Reduce time spent on documentation creation by 90\%
	\item
	Improve documentation quality by 20\%
	\item
	Increase developer satisfaction with documentation process by 50\%
\end{itemize}

Each of these metrics is directly tied to the pain point which the engineers are facing, and thus by solving them, the engineers will be able to focus on more value-adding activities.

The metrics will be measured at the end of the SAIL framework using the methods outlined in the table below.

\begin{longtable}{@{}
		>{\raggedright\arraybackslash}p{0.4\linewidth}
		>{\raggedright\arraybackslash}p{0.6\linewidth}
		@{}}
	\toprule
	\textbf{Metric} & \textbf{Measurement Method} \\
	\midrule
	\endhead
	\bottomrule
	\endlastfoot
	
	Reduce time spent on documentation creation by 90\% &
	Developer time tracking survey. \\
	
	Improve documentation quality by 20\% &
	Documentation quality survey\\
	
	Increase developer satisfaction with documentation process by 50\% &
	Developer experience survey \\

\end{longtable}


\textbf{Technical Feasibility \& Data Assessment:}

The first initial step done by the engineers was the Technical Feasibility and Data Assessment check. The engineers have managed to identify that the best approach for this use case is to leverage an existing LLM model and fine-tune it on the internal codebase of the company. The process which they have chosen was utilising the Github Copilot technology which was already preapproved by the safety and security team of the company. The engineers have managed to identify that the codebase of the company is sufficient in order to fine-tune the model and thus the technical feasibility check was passed. 

In terms of the data assessment, the engineers have selected the most appropriate code base to start on, this being the Python scripts which were among the most edited and worked upon and thus would have the highest return on effort once our pilot becomes the standard.

In terms of creating the strategy for the deployment, the team did research on how to distribute this use case to the widest number of engineers at once, and this is done via a VSCode extension. The skill to create it was not currently in the team and this was identified as a gap which needed to be filled before proceeding to the next stage.

\textbf{Pilot Design:}

The pilot stage for this use case involved creation of the smallest deliverable product which other engineers are able to use and gain immediate value. This MVP was a VSCode extension which is able to generate code documentation for python scripts. The process of creating this MVP had many challenges such as the lack of experience in creating VSCode extensions which had to be tackled by the engineers as an immediate learning curve and the first process of trial and error. 

The architecture of this solution was designed as an integration of GitHub Copilot in an agentic model with an underlying Claude Sonnet 4.5 model. This combination will be then enhanced with the special usage of markdown files whose purpose is to be guiding the Agentic model to the best efficiency possible. 

The sole purpose of the markdown files is to specify custom rules in terms of documentation, so that the model is then able to retrieve these files and know what the current standard of the internal rules and configurations is. 

This approach gives the model an actual leverage and a touch of customization over just simply using the out of the box Github Copilot functionality.


By the end of the pilot design stage, a clear roadmap was created serving as the backbone of the entire use case. This roadmap consisted of all the steps needed to be done from the onboarding until the final deployment, release, and rollout of this use case. This roadmap does not only serve the engineers to track the time inside of it, but also shows the management the complexity of the use case and gives them the ability to plan this use case over a period of time in a more structured way.

Once all of this was created, we were then able to move into the next stage. 

\textbf{Pilot Execution:}

The pilot execution stage for this use case started with the process of the actual onboarding and learning about how to successfully create and deploy a VSCode extension. Online resources were gathered mostly from the actual Microsoft services to understand how engineers are able to achieve this goal. We opted for the route of onboarding only one engineer instead of three because in the meantime the other two engineers had already started writing specific markdown files which would then guide the AI model in the agentic mode. 

By having this distributed approach, we are able to not be slowed down by the learning curve of creating this VSCode extension and continue working on the use case. 

The pilot execution stage lasted for around 3 weeks and by the end of it a working MVP was created which was able to generate code documentation for python scripts. The MVP was then tested by a small group of engineers and feedback was collected in order to improve the solution.

After the successful and improved MVP, the team was then able to move into the next stage of evaluation and scaling decision.

\textbf{Evaluation \& Scaling Decision:}

The process of evaluation and the scaling decision for this use case involved the actual analysis of the results and the interconnection between various entities such as the sponsor, the management team, and the security team.

The analysis looked at our success criteria which were outlined in the strategic alignment stage and evaluated whether they were met or not.

The numbers which we gathered from the documentation creation process involved metrics of reduction in the time spent on creation of documentation, improvement in documentation quality, and the overall satisfaction. 	

In the pilot stage, we managed to get the time spent on creation of the documentation reduced by 70\%.
The documentation quality was improved by 15\% and the overall satisfaction of the engineers with the documentation process was increased by 40\%.

Based on these metrics, we can clearly see that the MVP has a very strong potential and is already able to yield very promising results. 
As these results are already on the track to be successful, we are going to proceed with the creation of the final product and scaling it. 

If the metrics were not on track in terms of the promised and expected numbers, we would have to either reevaluate whether the targets which we set were too high or whether the observed metrics are objectively too low.

\textbf{Scale Deployment:}
The next step in the SAIL framework is the stage of scale deployment. This stage involves the actual rollout of the AI solution to a larger audience and collecting immediate feedback.
During the scale deployment phase, the engineers successfully managed to tweak the markdown files to be more appropriate for the code documentation while still collecting very valuable feedback from the engineers. This has created a cycle of improvement, leading into a better and more rigid solution. 

The process of deployment has been successfully achieved by establishing a private VS Code marketplace which is self-hosted and where every single AI use case developed internally is able to be deployed. This marketplace isolates the contact from outside world and therefore ensures security and safety of the internal tools. 

As every single engineer has the ability to simply download this VS Code extension inside their VS Code editor, the friction between installation and usage is completely minimized, thus improving the process of change management and reducing the resistance to change.

\textbf{AI Native Optimization:}
The final stage in the SAIL framework is the stage of AI native optimization. This stage involves the continuous improvement and optimization of the AI solution.

As the previous stage has been more focused on the distribution, the next stage needs to be focused on the actual improvement of the solution. Inside the company, we positioned our team as a center point of AI and allowed engineers to comment on the repository of this use case. By collecting this valuable feedback, we were able to improve the solution and tweak it according to the developers' preferences. By having a wider audience of users using the solution, we were able to yield very promising results and succeed in the process of continuous improvement.

After successfully implementing all the suggestions and comments, we were able to close the initiative as successful and move on to the next use case with the following results highlighted in the table below.

\begin{longtable}{@{}
		>{\raggedright\arraybackslash}p{0.4\linewidth}
		>{\raggedright\arraybackslash}p{0.6\linewidth}
		@{}}
	\toprule
	\textbf{Metric} & \textbf{Result Achieved} \\
	\midrule
	\endhead
	\bottomrule
	\endlastfoot
	
	Reduce time spent on documentation creation by 90\% &
	Achieved a 90\% reduction in time spent on documentation creation. \\
	
	Improve documentation quality by 20\% &
	Improved documentation quality by 22\% based on developer surveys.\\
	
	Increase developer satisfaction with documentation process by 50\% &
	Increased developer satisfaction by 55\% according to feedback forms. \\

\end{longtable}


The process of optimization does not, however, end at simply receiving the results of the survey. The process of continuous improvement is a never-ending story unless the use case diminishes in terms of the usage by the engineers. For this use case, a set amount of questions and remarks were still added even after several months of usage. This means that the process of continuous improvement is still ongoing, however, we are getting a value from a product which is already implemented and requires close to minimal support and maintenance. This example clearly demonstrates us that it is possible to create an AI product which is delivering sustainable value. In order for the highest amount of engineers to properly use this tool, a set of workshops and training sessions were created in order to onboard the engineers and show them the value of this tool. By having this change management process in place, the adoption of the tool was significantly increased and thus the overall impact of the use case was maximized.

\subsection{AI-driven Refactoring}\label{ai-driven-refactoring}
\textbf{Entry Point:}

The use case to be explored and implemented is the process of developing an AI-native agent pilot which will be responsible for refactoring any part of the existing code base. This use case is going to be also based on GitHub Copilot with Claude Sonnet 4.5 and a set of markdown files guiding the agentic model on how to refactor the code. 

The process of refactoring is extremely important and needs to have a high level of quality and a concrete set of rules as each engineer views what is the best for the company and the code differently. This brings many ideas to the table, but does not determine what is the best approach for refactoring. Certain engineers prefer verbosity for the sake of readability and others prefer conciseness for the sake of efficiency. This conflict of interest does not have a concrete result and a concrete base. For a concrete decision point to be done, a set of rules needs to be evaluated, created, and enforced. This is exactly where the agentic AI model comes into play. 

By having a set of markdown files and guiding the AI agentic model to enforce a strict refactoring, we are able to eliminate this problem between two engineers disagreeing on what is the best approach for refactoring. 

\textbf{Opportunity Scouting:}
This opportunity of having an AI-driven refactoring comes from the need to have a very high level of code which is being monitored by various assessments. Assessments such as: AST analysis, Cyclomatic complexity, Code duplication and other code quality metrics are extremely important for the overall health of the codebase and thus having a tool which is able to automatically refactor the code based on these metrics is extremely valuable.

The opportunity to develop an internal AI-driven tool which is able to refactor any type of code base based on the set of well-maintained markdown files used for configuration brings the company, on the first glance, an immediate return, not only from the perspective of having a clear, well-maintained and readable code, but also having long-term clean code for the new engineers to be able to onboard themselves on it and bring value in a shortened amount of time.

\textbf{Strategic Alignment:} 
The process of strategic alignment for this use case was in fact very simple, because once the management was aware of the problem that many engineers have many different coding standards, having a unified approach to this problem was extremely valuable. 

The main approach when negotiating with the upper management focused on the long-term value of having a clean codebase rather than focusing on the fact that it's done by AI due to the novelty. This approach only enforced our integrity and seriousness of the use case.

The vision of having a unified approach to declutter the current processes and help engineers have a more readable code while still preserving the efficiency of it was extremely valuable and thus the management was on board with this initiative from the very beginning.

In order for maximum efficiency, a team of two people were selected to tackle this use case. Due to the nature of the company and the trust which the AI community of practice had from the upper-level management, conveying the message of selecting just two engineers from the given team was not a problem that needed to be tackled as upper management stood on our side and trusted our judgment.

\textbf{Technical Feasibility \& Data Assessment:}
From the perspective of technical feasibility and data assessment, we had to decompose this problem. 
The first step was to identify whether there is already a set of rules which the company is currently refactoring by. The second question was whether the code base was already in a sub-par state and the use case even had the need to be created. This would mean that if the code base was ordered in an amazing and perfect and clean state according to the latest coding standards, the need for this use case would be completely abolished. 

After further evaluating the current code base, the conclusion came to it being in a sub-par state and the need for this use case became apparent. In terms of the rules and standards of refactoring throughout the entire company, the only existing rules were the ones enforced by the PEP8 standard developed under Python. This however, shall not be confused with refactoring, PEP 8 is a linting standard and the vision of our AI agent is concrete refactoring.

As this use case will be once again developed under GitHub Copilot with the already existing LLM model Claude Sonnet 4.5, the technical feasibility was already passed as the safety and security team have already preapproved this technology for internal usage.


When it comes to estimating the impact and expectations from this use case, the team outlined the following metrics:
\begin{itemize}
	\tightlist
	\item
	Reduce code complexity by 25\%
	\item
	Improve code readability by 30\%
	\item
	Increase developer satisfaction with refactoring process by 40\%
\end{itemize}

The metrics outlined closely describe what the engineers are struggling with and what the main problem is which they are facing. By solving these problems, the process of refactoring will become extremely streamlined and the engineers can focus on other tasks which are more value-adding. The metrics will be measured at the end of the SAIL framework using the methods outlined in the table below.

\begin{longtable}{@{}
		>{\raggedright\arraybackslash}p{0.4\linewidth}
		>{\raggedright\arraybackslash}p{0.6\linewidth}
		@{}}
	\toprule
	\textbf{Metric} & \textbf{Measurement Method} \\
	\midrule
	\endhead
	\bottomrule
	\endlastfoot
	
	Reduce code complexity by 25\% &
	Junior Engineers Survey \\
	
	Improve code readability by 30\% &
	Code review surveys and readability scoring tools.\\
	
	Increase developer satisfaction with refactoring process by 40\% &
	Developer experience survey \\

\end{longtable}

\textbf{Pilot Design:}

The Pilot Design stage started with research from the team on the best practices when it comes to refactoring. The latest and greatest standards were gathered, documented, and their input was written in the markdown files. The focus of the team was purely Python, as it's the most dominant and impactful language in the company. By having clearly written rules which are enforced, we are able to ensure that the code quality is constantly improved and the engineers are able to focus on more value-adding activities.

After thorough research, the team decided to proceed to implement their own GitHub Copilot agent based on Claude Sonnet 4.5 as the main driver. 

The designed process of how to use this tool was created and came to the following steps:

1. Engineer selects the code segment to be refactored.
2. Engineer activates the AI refactoring tool via VSCode extension.
3. The tool retrieves the relevant markdown configuration files.
4. The AI agent analyzes the selected code segment based on the rules specified in the markdown files
5. The AI agent generates refactored code suggestions.
6. The engineer reviews the suggestions and accepts or modifies them.


By having a very straightforward process when it comes to usage of the tool, the friction for adoption is minimized. 
The vision is to then package this functionality inside the VS Code extension and simply launch it on other engineers' computers. 
This clearly defined architecture and process standard then led the team to the next stage of pilot execution.

\textbf{Pilot Execution:}

The pilot execution stage for this use case started with a rigorous process of following the roadmap outlined in the step before. From the beginning of the entire roadmap, the goal was to implement an MVP which fulfills the objective of refactoring Python code based on the input and standards from the markdown files.
The two engineers tackled this use case over the span of three months and successfully implemented all of the tickets outlined in the roadmap and developed the first version of the MVP.
Once the MVP was done, a presentation to showcase the first value creation was done in the company to start the internal process of marketing and awareness. By having the first MVP presented to the wider audience, the team was able to collect valuable feedback from the engineers and improve the solution even further.

After the successful improvement based on the presentation, the MVP was then forwarded to another team which conducted rigorous testing and gave a series of comments that helped the team improve the solution and refine it according to the engineers' preferences.

The most valuable feedback was to have a tiered approach when it comes to the refactoring levels. This means that the engineers wanted to have the ability to select whether they want a light refactoring, medium refactoring, or a heavy refactoring. This feedback was extremely valuable as it allowed the team to improve the solution and make it more flexible to the engineers' preferences.

Another very valuable feedback was to implement team-specific markdown files. This means that each team is able to have their own set of rules and standards when it comes to refactoring. This, on the first glance, might seem counterproductive, but on the other side a solution needed to be found.

The new feedback came from the principle that the whole company cannot sustainably live off a single coding standard due to different customer programs being delivered to different customers, each requiring completely different requirements and standards, and therefore a more modular approach came into the discussion and further implementation. 

All the feedback gathered in their data collection stage was then implemented and the team was able to move into the next stage of evaluation and scaling decision.

\textbf{Evaluation \& Scaling Decision:}

Once the process of evaluation and scaling decision came to place, the team had already gathered trust of management, the engineers, and had already received company-wide recognition. This has outperformed the existing expectations from the beginning of simply keeping it as an MVP and thus the decision to proceed, reject or iterate was more than clear. The need for this use case was clearly in the company and it immediately solved another pain point which the engineers were facing.

Now the correct question would be to ask whether this is not enough and if any extra effort put into this use case would yield a positive return. On one side, the solution was good enough as it was already delivering value and engineers started using it, but on the other side, the team saw a potential for further improvement and thus the decision to proceed with scaling and deployment was made. This improvement came to place mostly in the aspect of having more languages supported rather than just Python. By having more languages supported the overall impact of the use case would be significantly increased and thus the decision to proceed with scaling and deployment was made.


From the perspective of metrics, the following numbers were achieved during the pilot stage. 
\begin{longtable}{@{}
		>{\raggedright\arraybackslash}p{0.4\linewidth}
		>{\raggedright\arraybackslash}p{0.6\linewidth}
		@{}}
	\toprule
	\textbf{Metric} & \textbf{Result Achieved} \\
	\midrule
	\endhead
	\bottomrule
	\endlastfoot
	
	Reduce code complexity by 25\% &
	Junior Engineers Survey indicated a 30\% reduction in code complexity. \\
	
	Improve code readability by 30\% &
	Improved code readability by 40\% based on developer surveys.\\
	
	Increase developer satisfaction with refactoring process by 40\% &
	Increased developer satisfaction by 30\% according to feedback forms. \\
\end{longtable}

\textbf{Scale Deployment:}
As the decision to scale and deploy the solution was made, the team started the process of preparing the solution for the wider audience. While entering into the stage, the solution was overly packaged as a VS Code extension ready to be deployed. The main challenge which the team was facing is the lack of supported languages and therefore the same logic was implemented for further languages being: (Shell Script, C++ and C).

The process of scaling the solution into more languages could unfortunately not be reused from the current status done in Python as new research needed to be conducted. The reason for this is that each language has its own set of standards and rules when it comes to refactoring, and therefore the team needed to go through the same process of research, documentation, and implementation for each language separately.

After successfully implementing the additional languages, the team could then proceed with the deployment of the solution to a larger audience of engineers. This was achieved by once again utilizing the private marketplace hosted inside the company. The engineers could simply download the VS Code extension from the marketplace and use it immediately inside their editors. By simply having this option the friction for adoption was minimized and the team was one step closer to the final stage of AI native optimization.

This however does not paint the whole picture. Only having the use case available to the engineers does not change the behavior and the current mentality of them. Many of the software engineers didn't even know that the marketplace offered such a great tool. This had to be done via rigorous training, workshops, and internal company marketing to convey the message that there exists a tool and this tool shall be utilized in order to gain significant improvements in terms of time, code quality, and a common vision for great code standards. 


\textbf{AI Native Optimization:}
The final stage of the SAIL framework is the stage of AI native optimization. Throughout the stage, the team gathered many more comments regarding the code refactoring, the latest changes, the standards, architecture, and principles which the engineers would like to see implemented. The larger audience once again allowed for better final outcome. We have utilized the principle of having a public repository, where engineers could simply comment their requests and suggestions which they would like to see and thus giving people the power to control the direction of the use case.

By having this continuous improvement process in place, the team was able to successfully implement the most requested features and suggestions and thus improve the solution even further.

The ongoing iterative optimization process wasn't so active as expected. The external engineers were initially very proactive when it came to suggesting new standards, but once a common ground was achieved only very few people were commenting on changes. 
This gives us a conclusion that  the solution has reached a point of maturity and the engineers are satisfied with the current state of it.

This claim can be backed by the final survey done among the engineers which yielded the following results:
\begin{longtable}{@{}
		>{\raggedright\arraybackslash}p{0.4\linewidth}
		>{\raggedright\arraybackslash}p{0.6\linewidth}
		@{}}
	\toprule
	\textbf{Metric} & \textbf{Result Achieved} \\
	\midrule
	\endhead
	\bottomrule
	\endlastfoot
	
	Reduce code complexity by 25\% &
	Junior Engineers Survey indicated a 28\% reduction in code complexity. \\
	
	Improve code readability by 30\% &
	Improved code readability by 35\% based on developer surveys.\\
	
	Increase developer satisfaction with refactoring process by 40\% &
	Increased developer satisfaction by 45\% according to feedback forms. \\
\end{longtable}


The final survey clearly tells us that the solution was implemented in a fashion which was tailored not only to the general engineers but also to the juniors. The juniors are usually the engineers with the least amount of experience with code, and by tailoring also to their understanding, we are able to ensure that the solution is sustainable in the long term as the new hires will also be able to utilize it. 

If we only extrapolate the fact that juniors preferred this tool, we might come to a conclusion this tool makes code very inefficient and very verbose. However, this is not the case as another global survey was sent to all engineers and the result of 45\% satisfaction increase clearly indicates that the solution is well balanced. If the verbosity would be increased and thus compromising on the efficiency, senior engineers would report this use case in the survey as underperforming and thus giving us a clear indication that the solution is not well balanced.

Therefore, we can strongly conclude that the use case is well balanced between the verbosity so the juniors understand and the efficiency so the seniors are also satisfied.


\subsection{Pull Request Review
	Automation}\label{pull-request-review-automation}

\textbf{Entry Point:} 

The following use case of having an automated pull request review system is extremely valuable for any software organization. Every single line of code just added inside of the code base of an organization needs to go through a pull request review. This means that a human is usually reviewing code coded by another human. No matter the size of the organization, this process is extremely time consuming and tedious. Oftentimes, junior engineers are creating pull requests which are sub-par to the quality of the code inside of the code base and senior engineers are usually tasked with reviewing these pull requests. This creates a bottleneck as senior engineers are usually the busiest people inside of the organization and having them review code which is not up to the standards is a waste of their time. By having this process automated by an AI agent, senior engineers are able to save expensive precious time to work on more critical features and functionalities, rather than reviewing code which could be done by an AI agent.

An AI agent which is able to review pull requests based on a set of markdown files guiding it on what are the standards and principles of the company is extremely valuable. This agent will be able to review the code, give comments, suggest improvements and even approve or reject the pull request based on the quality of the code. This will create a significant time saving for senior engineers and thus allowing them to focus on more value adding activities.

The benefit of having an AI agent running on a virtual machine is that it does not only see the static content of the pull request and review it from the perspective of linting and organization of the code, it is also able to execute the code as it has resources dedicated for this purpose.

This gives the AI agent a significant advantage over human reviewers as it is able to test the code, run unit tests and integration tests and thus give a more holistic review of the pull request.

\textbf{Opportunity Scouting:}

The phase of opportunity scouting involves identification of the pain point, this being the pull request review process. This use case was pitched by senior engineers as they are the ones responsible for reviewing the pull requests and keeping the quality of the code base high. Once the company has an AI-driven pull request review automation, it can restructure the entire process of code review and allow senior engineers to focus on more value-adding activities. The time saved from this process is extremely valuable as senior engineers are usually the most expensive resources in the organization and having them review code which could be done by an AI agent is a waste of their time.

\textbf{Strategic Alignment:}
The complexity of this use case is greater than the already existing ones. This means that the strategic alignment process needs to be done with multiple entities. This use case is not focused on the alignment of the vision, mission, and the goal of the company; this use case comes to life because it can revolutionize how the company operates. The key strategic entities which the team will have to align with are the IT department, the Vice President of Engineering, and the Safety and Security Department. 

As this use case requires more manpower, a team of four engineers was assembled. The team consisted of two senior engineers, one junior engineer, and one regular professional. This gave the team the ability to tackle the complexity of the use case from multiple angles and ensure that the solution is well rounded and robust.

The strategy alignment on this use case was done via a series of meetings with the key stakeholders. The first step was to present the use case to the IT department and get their buy-in. The second step was to present the use case to the Vice President of Engineering and get his buy-in and initial tips while advancing on the roadmap. The third step was to present the use case to the Safety and Security team and verify what type of infrastructure is needed in order to ensure that the solution is compliant with the company's security standards.

All of these steps are very crucial to ensure that the use case is well researched and aware of all the potential roadblocks down the line. If all the strategic points wouldn't be properly aligned, any factor could totally disintegrate the entire idea. This would mean that all the effort put into the use case would be wasted, and thus the strategic alignment is extremely crucial for the success of this use case.

\textbf{Technical Feasibility \& Data Assessment:}
From the perspective of the technical feasibility and data assessment, the team had to first analyze if the already existing infrastructure is sufficient for an AI agent to be hosted on. The team had to also identify whether it is even possible to create API calls between the current infrastructure and the used services to fulfill the use case of reviewing pull requests with an AI agent.
The technical feasibility check was conducted very closely with the IT department. 

The final conclusion from the IT department was that all the services and our approach is safe and secure and the only need was to deploy a separate virtual machine acting as a server on which the AI agent will be hosted and run. This virtual machine will be connected via API calls to the existing code repository and thus will be able to review pull requests in close to real time.

The data assessment part involved identification of the existing code base and the quality of it. The team had to ensure that the code base is sufficient in order to train and fine-tune the AI agent to be able to review pull requests effectively. The team identified that the code base is indeed sufficient and thus the technical feasibility and data assessment check was passed.

In order for the team to try to aim for certain benchmarks, the following metrics were established as our success criteria. 

\begin{itemize}
	\tightlist
	\item
	Achieve an automated pull request review 
	\item
	Reduce the time spent on pull request reviews by 100\%
	\item
	Increase developer satisfaction with pull request review process by 50\%
\end{itemize}

This time one of the metrics does not have a clear number next to it, but it is rather a yes or no observation. The two other metrics are clearly defined in terms of the optimization. It is very important to point out that the first metric, to achieve an automated pull request review, is a remarkable achievement already on its own and thus the other two metrics are more of a cherry on top. Even though on the first glance the first metric might not be useful for the observer, any type of automated pull request review already brings value in terms of saving time for senior engineers. In order for us to understand whether the use case was successful or not, the team will conduct measurements at the end of the SAIL framework using the following methods. 

\begin{longtable}{@{}
		>{\raggedright\arraybackslash}p{0.4\linewidth}
		>{\raggedright\arraybackslash}p{0.6\linewidth}
		@{}}
	\toprule
	\textbf{Metric} & \textbf{Measurement Method} \\
	\midrule
	\endhead
	\bottomrule
	\endlastfoot
	
	Achieve an automated pull request review &
	System logs and monitoring \\
	
	Reduce the time spent on pull request reviews by 100\% &
	No pull request reviews anymore \\
	
	Increase developer satisfaction with pull request review process by 50\% &
	Developer experience survey \\
\end{longtable}

\textbf{Pilot Design:}


The pilot design stage for this use case involved a series of investigations, research, and planning. The first step was to identify that in order to have an AI agent review pull requests, a virtual machine acting as a server is a precondition.
This virtual machine will be hosting our AI agent which will be directly communicated via API calls to fetch the latest changes and be able to execute them independently on the code base which it will have in its current workspace. 

The backbone of our AI agent will be once again GitHub Copilot with Claude Sonnet 4.5 as the main LLM model. This model has already successfully proven itself in the previous use cases even though its complexity was not up to the level of reviewing pull requests and executing code on human behalf.

The next step in the pilot design stage was to create a set of markdown files guiding the AI agent on what are the coding standards and principles of the company. This is extremely crucial as the AI agent needs to understand what is expected from it in order to review pull requests effectively. The markdown files will be constantly updated and improved based on feedback from senior engineers and changes in coding standards.

The next step in terms of our pilot design stage was the creation of a set of markdown files which are going to be guiding the AI agent and helping it understand the dependencies between files, functions, and logical principles. This is done not because AI is not able to detect patterns and act upon them, but because for every single execution, AI doesn't have to do this tedious process and it can already use the existing knowledge base. The reasoning for this action is that it should significantly reduce the time required for the AI agent to review pull requests and thus improve the overall efficiency of the solution. 

The following step will require an establishment between the virtual machine hosting the AI agent and the code repository via API calls. This will allow the AI agent to fetch the latest pull requests and review them in close to real time.

The step which will determine what the success of the pilot design is is the definition of the actions which the AI agent will be able to take. As already mentioned in the entry point, the AI agent will be able to review the code, give comments, suggest improvements, and even approve or reject the pull request based on the quality of the code. This will create a significant time saving for senior engineers and thus allow them to focus on more value-adding activities. This will have to be marked down and will act as the core identity of the AI agent. 

In the final step, the pilot will be tested on a sample of pull requests and gather feedback from senior engineers. This feedback will be crucial in order to refine the AI agent and improve its performance. This use case will rely on the feedback loop from senior engineers more than any other use case as we are losing the sense of human review and thus the trust from senior engineers needs to be built up in order for the solution to be successful.  


After establishing all the required components and architecture for successful execution of the pilot, aligning the strategy with the key stakeholders, and passing the decision whether the use case is technically feasible or not, the team was ready to proceed with the pilot execution stage. 

\textbf{Pilot Execution:}
	
	The pilot execution stage started with an unexpected challenge. This challenge involved restriction of API access and configuration of the firewall requiring tedious process of approval from the IT department and the safety and security department. The reasoning for this restriction was the security concerns regarding the access of an external virtual machine to the internal code repository. 

	After successful implementation of the agent, the procedure to implement the markdown files acting as a knowledge base for the AI agent began. This process involved thorough research and documentation of the coding standards and principles of the company. The team had to ensure that the markdown files were comprehensive and covered all the necessary aspects of code review.
	

	This, however, did not end up being sufficient as during the initial testing phase, several senior engineers discovered that the AI agent was approving syntactically correct but architecturally flawed and insufficient code.
	Due to this issue, the team was prompted to do further refinement, deeper analysis, and alignment sessions with the VP of Engineering to guide them towards a successful path. During these sessions, a proper strategy was developed for the AI to understand the architecture, not only on the principle of syntax but also under the principles of design and architecture. 

	Another flaw which the junior engineer identified was a very critical bug where an agent failed to detect circular dependency across multi-file pull requests. This led to the implementation of a multi-pass review architecture which analyzes the full repository context before approving changes. Having an approach such as this significantly improved the quality of the pull request reviews and thus increasing the trust from senior engineers.

	During the execution phase, senior engineers discovered a harsh truth from the principle of change management: the trust.
	This lack of trust came from the fact that the agent's review comments lacked contextual reasoning explanations and this prompted the team to implement a verbose logging mode where the AI justifies every rejection with specific markdown guideline references. 
	This further justification improved the perceived trust and helped engineers understand the reasoning behind the rejections.
	
	A very positive aspect of this execution phase, was the unexpected win which the team achieved. This win being the instance in which the agent discovered a legacy code pattern violation that human reviewers had been consistently missing for months. This discovery validated the use case to previously skeptical VP of Engineering and secured additional resources for scaling beyond pilot phase.

	The additional resources beyond the pilot phase were extremely crucial as they will serve us as a quality gate on whether to proceed with scaling and deployment or to iterate further. This however shall not be the same determining factor as metrics and impact will be guiding us in the next stage of evaluation and scaling decision.

\textbf{Evaluation \& Scaling Decision:}
	As already mentioned, the impact and the trust gained from the VP of Engineering gave not only the team but the engineers already testing this pilot the confidence that the solution is indeed valuable and thus the decision whether to proceed with scaling and deployment or to iterate further was not as complex as in other use cases.

	In order to follow the SAIL framework correctly, the team had to evaluate the impact of the pilot and measure the success criteria outlined in the technical feasibility and data assessment stage.
	
	The first metric which was outlined was the achievement of an automated pull request review. As this metric was more of a yes or no observation and the source of truth was the system logs and monitoring, the team can confirm that a fully automated agent reviewing pull requests is indeed achieved.

	The second metric involves the reduction of time spent on reviewing and monitoring pull requests by 100\%.
	The vision was that every single pull request opened in the company would be left to the AI agent and that humans would not be in touch with reviewing the code anymore. This, however, has not become the case. In order to set the reality straight, the engineers are still overseeing the pull request review process and thus the time spent on pull request reviews is not reduced by 100\%. This metric shall, however, be questioned, whether the reason is the incompetence of the AI agent or the lack of trust from senior engineers. It is very natural that humans do not trust an AI agent straight out of the box. The process of change management requires time, results and proven track record of success in order for humans to trust the AI agent. What, however, has to be mentioned is the significant decrease in terms of the actual time spent on pull request reviews by senior engineers. 

	In the current state, the AI is able to review the pull request based on the markdown files and give comments and suggestions. The senior engineers are only overseeing the process and giving the final approval in terms of whether the code in the pull request truly fulfills the quality standards and its purpose matches the requirements outlined in the ticket. Based on the feedback from senior engineers, the time spent on pull request reviews has been reduced by 70\%, which is a significant achievement on its own. In this case, the expectation of reducing the time spent on pull request reviews by 100\% was not achieved, but it could be argued that the original expectation was too high and the achieved result is still extremely valuable.


	The last metric involved the increase of developer satisfaction with our new pull request review process by 50\%. This metric was measured by an actual developer survey. The results from the survey indicated an increase in developer satisfaction by 40\%. Once again, the final metric did not meet the original expectation and once again it would be appropriate to question whether the original expectation was too high.

	If we simply extrapolate the increase in developer satisfaction by 40\% and looked at the number from a birds eye view, we can clearly see that this is a significant achievement. What still has to be questioned is how can we improve this? 

	As in the current stage the use case is still in its early times and is by far not perfect, there is a lot of room for improvement. 

	The current potential improvements which the team has identified are mostly related to improvement of logging and dynamic tracking of the changes. By having a more dynamic tracking of the changes, the AI agent will be able to understand the context of the pull request better and thus give more accurate reviews.

	After the final evaluation of all the metrics and an overall impact of the pilot, the team has come to a conclusion that the use case is indeed valuable and it will be the right decision to proceed with the process of scaling and deployment rather than iterating further to improve the pilot. 

\textbf{Scale Deployment:}
The process of scaling this use case came down to two main divisions: the technical scaling and the organizational scaling. From the perspective of technical scaling, the team had to simply ensure that the agent will be able to handle the load of the entire organization rather than just a small subset of engineers testing the pilot. This was achieved by scaling vertically the virtual machine hosting the AI agent and thus ensuring that it has enough resources to handle the load. 

The larger unknown variable came from the perspective of organizational scaling due to the change management process of having to abandon the process of review and having humans review the code. The management process involved a series of hackathons, workshops, and solution demos during which the usage and its benefits were presented to the wider audience. By having this rigorous change management process, the team were successfully onboarded and trust was gained. It is very important to mention that the whole organization did not digest the change at once. The process of scaling was going from team to group to department to division and finally to the entire organization.   

The process of scale and deployment was heavily supported by the upper management, especially the VP of Engineering. This support was extremely crucial to gain trust for the engineers as they were more likely to trust the solution if it was endorsed by the upper management. In terms of the timelines, the aspect of technical deployment was achieved almost instantly due to the already existing infrastructure. The organizational deployment, however, took around three months as the change management process is always a time-consuming process.

Once the solution was successfully scaled, deployed, and people were onboarded, it was time to decide whether we are done with the use case or if we want to give it further iterative optimization and continuous improvement in the next stage.

The decision came to the reasoning that the use case is indeed valuable and the state of the pull request review automation is not perfect yet. New requirements and feedback are constantly coming in, and this gives us a clear indication that the use case is not finished yet and will still require further iterative optimization and continuous improvement in the next stage of AI-native optimization.

\textbf{AI Native Optimization:}
The stage of the SAIL framework which concludes the entire process is the AI native optimization. After the use case being successfully designed, analyzed, refined, implemented, scaled, rolled out, it's time to maintain it, optimize it, improve it further, and yield continuous and sustainable value over time. The capital investment in terms of time and resources has already been made and therefore the best option for the team was to try to ensure that the use case meets the current requirements of the engineers and provides value in terms of the pull request review process. The question of what the engineers want and expect can only be measured by one way and it's talking to the customers, in this case the engineers, as this tool is only internal. 

Based on the survey done several months after the deployment of the solution, the team gathered more metrics from the already outlined methods.

\begin{longtable}{@{}
		>{\raggedright\arraybackslash}p{0.4\linewidth}
		>{\raggedright\arraybackslash}p{0.6\linewidth}
		@{}}
	\toprule
	\textbf{Metric} & \textbf{Result Achieved} \\
	\midrule
	\endhead
	\bottomrule
	\endlastfoot
	
	Achieve an automated pull request review &
	System logs and monitoring confirm continuous operation \\
	
	Reduce the time spent on pull request reviews by 100\% &
	Time spent on pull request reviews reduced by 75\% \\
	
	Increase developer satisfaction with pull request review process by 50\% &
	Developer satisfaction increased by 55\% \\
\end{longtable}

Based on the gathered numbers in the AI-native optimization step, we can clearly see a difference between the stage of pilot execution and the AI-native optimization. This being mainly in the last metric in terms of the developer satisfaction. 
The reasoning behind increased developer satisfaction is the fact that the team was able to implement further improvements based on the feedback gathered from engineers and thus fulfilling their needs better. It can also be noticed that the final metric has actually surpassed the original expectations, which is a remarkable achievement on its own. 
The final metric tells us that two out of the three expected metrics have successfully passed our initial threshold.
As already mentioned in the pilot execution phase, the metric regarding the time spent on pull request reviews being reduced by 100\% was always a very ambitious goal and therefore the achieved result of 75\% is still extremely valuable.


It is also very important to note that this use case will stay here, but hopefully the maintenance will not. The use case should reach its maturity, meaning the point during which engineers are satisfied with the functionality and performance of the solution and therefore the need for further improvements will be minimal. Important to mention, down the line there should be a team maintaining the solution and understand the codebase as well as the architecture behind it. 

The team is planned to stay in the stage of AI native optimization as long as the demand for changes and improvements is present. Once the team reaches the point of diminishing returns and the engineers are satisfied with the solution, the team will be disbanded and the maintenance will be handed over to a smaller team responsible for maintaining the solution.



\subsection{Internal AI Chatbot for Employee
	Queries}\label{internal-ai-chatbot-for-employee-queries}

\textbf{Entry Point:} The following use case focuses on a very important and crucial aspect of any organization, being the internal communication and knowledge sharing. The distributed knowledge and lack of well-connected documentation is an issue troubling most of the organizations in software industry. This problem will be solved by ad internal LLM based chatbot with capability of answering questions and providing relevant information in close to real time. This chatbot does not only answer questions per page, but is capable of creating an internal interconnected mind map of knowledge which contains every single detail the company decides to put in it and allow for interal search and knowledge sharing. This capability of restricting in what to add in the knowledge base is crucial as the company is able to restrict any sensitive information, which shal not be availabe for other employees. The chatbot will be able to answer questions regarding internal tools, processes, status of projects and other relevant information which the engineers might be questioning. This chatbot will be able to save time and increase productivity of the engineers as they will not have to search for relevant information and the access to information at any-time and thus reducing the load on employees with knowledge silos.

\textbf{Application of SAIL:} - \textbf{Opportunity Scouting:} The opportunity scouting phase for this use case will involve identification of the painpoint of knowledge sharing and internal communication. The entity most likely coming up with this idea is the Junior Engineers and new hires, which have a tough time during onboarding as a new environment in a poorly structured company document-wise. The average time of engineers trying to search for relevant documentation is around 30 minutes per day, which translates to 10 hours per month per engineer. For a team of 1000 engineers this translates to 10000 hours per month of time wasted on searching for relevant documentation. The potential monetary value of this time is around EUR 800,000 per month (assuming EUR 80/hour fully loaded cost). The time and money wasted on searching for relevant information is a very significant amount, which bleads through the hands of the company.

\begin{itemize}
	\tightlist
	\item
	\textbf{Strategic Alignment:} During the strategic alignment phase the
	team will have to ensure that the initiative alligns with the
	strategic goals and OKRs of the company. The key strategic goals and
	OKRs in this case could be
	\item
	Reduce time spent searching for information by 50\%
	\item
	Improve employee satisfaction score regarding access to information by
	20\%
	\item
	Increase onboarding satisfaction score by 15\%
\end{itemize}

These objectives and key results represent an internal improvement of the companies processes. Not nescessarily a direct revenue generating initiative, however the impact on the internal processes and efficiency of the workflows creates time and space for more value adding activities. This indirect value creation is a process which is often overlooked and therefore might create friction in terms of buy-in from management and governance board. However, with correct framing of the value proposition and focus on the time savings and generation of space of engineers to do more value adding activities during this time shall be the priority when negotiating the buy-in.

\begin{itemize}
	\item
	\textbf{Technical Feasibility \& Data Assessment:} From the
	perspective of technical feasibility the teams will have to evaluate
	the connections and project management tools which they are using. The
	company has to realise their infrastructure assets in terms of server
	capacities both in RAM and Graphics cards as this solution will
	require an interal LLM to be hosted on premises. An option to
	outsource this processes opens up a lot of security concerns and
	therefore the company shall avoid this path. When focusing on
	appropriate technical feasibility the company needs to follow the
	outlined formula:
	\item
	Size of the company (numbers) * activity rate (\% of users which shall
	be using the tools at once) = Required capacity to host of users at
	once
\end{itemize}

If the company has 1000 employees and the activity rate is 10\% (100 users at once) the company will need to have a server capacity of around 80GB of RAM and 2 A100 GPUs in order to ensure smooth operation of the LLM based chatbot. The price of a new A100 GPU is around EUR 10,000 and the RAM is around EUR 500/16GB. This means that the initial investment for the hardware will be around EUR 25,000 which is a very reasonable price for the value it is able to deliver as the company bleeds through EUR 80,000 per month on searching for relevant information. The ROI of this investment is very high and the payback period is around 1 month including the price of employees working on the implementation of the solution.

\begin{itemize}
	\item
	\textbf{Pilot Design:} The stage of pilot design will involve
	outlining the steps required for the solution to come from an idea to
	first MVP which is usable for the end users. The first step shall be
	setting up the hardware and the LLM model and choosing the model which
	is the most suitable. The initial implementation shall be done with a
	smaller model like Llama 2 7B or Falcon 7B in order to ensure that the
	model is able to run on the hardware available. The second step shall
	be the data collection and cleaning, as the data quality is crucial
	for the performance of the model. The data shall be collected from
	internal documentations, either PDFs or tools like Confluence or
	Sharepoint. Once this data is gathered a process to create a vector
	database containing this data shall be created. The next step will be
	the integration of the LLM with the vector database and the creation
	of a simple UI which is able to query the LLM and provide relevant
	answers. The final step will be the testing and feedback collection
	from a small group of users in order to ensure that the solution is
	working as intended and is able to deliver value. The success metrics
	for this pilot will be:
	\item
	Reduction in time spent searching for information by 40\%
	\item
	User satisfaction score of 4/5 or higher
	\item
	\textbf{Pilot Execution:} From the perspective of pilot execution the
	team will be following the outlined map from the pilot design stage.
	The pilot design stage should take into consideration deviation from
	the plan due to unepexcted realms of the AI world. This stage does not
	only focus on implementation but on rapid review cycles and small
	testing by external users. The short demos shall be included for
	external teams to verify quality of the output during the
	implementation part as the most important metric of this
	implementation is the quality of output. Once engineers do not trust
	the output the adoption rate will be logarithim and will plateu quick.
	The pilot execution is expected to be delivered in 6 weeks as this
	gives enough time for the engineers to onboard to topic, setup
	infrastructure, collect data, clean data, train model, integrate model
	and test the solution. After 6 weeks it is not expecte that the full
	solution will be delivered but an MVP which fullfils the main
	requirements and meets the success criteria.
	\item
	\textbf{Evaluation \& Scaling Decision:} From the perspective on
	whether to scale the solution the main question is what does scaling
	mean? From the perspective of hardware the need to scale is out of the
	scope as either the company has to double in size or the amount of
	concurent users has to double, if neither of these metrics
	dramatically change the need to scale hardware-wise is not in the
	question. The question where to scale or not comes only when the MVP
	was delivered on a sample of the internal documentation and not the
	whole internal map was created. The decision to scale in that scenario
	involves inclusion of more data sources and more data into the vector
	database and retraining of the model. This stage than becomes very
	crucial as with more data the model is more likely to hallucinate and
	provide wrong answers as it might misunderstand concepts incorrectly
	written in the documentation. The decision to scale also involves the
	decision on adding new features such as inclusion of links to sources
	whic the AI chatboot is retrieve the information from. This feature is
	very important as it increases the transparency of the model and
	therefore increases the trust of the users in the output. The decision
	to scale still relies on the outcome of the metrics from the
	evaluations. These dependent metrics need to be fullfilled in order to
	proceed with the scaling decision.
	\item
	\textbf{Scale Deployment:} The scale deployment process will involve
	introduction of the new data and decision on which features to
	include. This proccess will be handled by close collaboration of the
	AI engineers and the end-customers (engineers) leaving feedback on the
	experience both from UI perspectivity and output quality. This process
	shall be on the quicker side as the pilot on its own does not change
	drastically but rather improves the quality of the output and the
	experience of the users.
	\item
	\textbf{AI-Native Optimization:} The continuou improvement and
	optimization will be done by retraining and optimising the internal
	LLM based chatbot which enhances the productivity of the engineers.
	Towards the end of the optimization phase new metrics will be
	calculated and the team will evaluate whether the solution is still
	delivering value or not. The goal of this stage will be to ensure that
	the internal LLM based chatbot has close to 100\% adoption within the
	engineering teams. Some tasks of continuous improvement could be:
	\item
	Refinement of response accuracy based on user feedback.
	\item
	Adding features such as source linking, multi-language support, and
	voice interaction.
\end{itemize}

\section{Summary of Findings}\label{summary-of-findings}
The outlined use cases are a perfect showcase of the immense potential of AI adoption in software companies. The structured approach which the SAIL framework provides allows teams and companies to tackle complex problems in a very manageable way and decompose them into smaller iterative steps. Each of the use cases outlined in this chapter demonstrates significant time and cost savings as well as improvements in productivity and employee satisfaction. As already mentioned, the use cases presented are not only theoretical but have been implemented in real-life scenarios and thus the numbers provided are based on real-life data and feedback from engineers. This elevates the credibility of the outlined use cases and thus provides a solid foundation for further exploration and implementation of similar use cases in other software companies.

In terms of the actual metrics, the following table lists the summary of the actual impact measured across all the outlined use cases:

\begin{longtable}{@{}
		>{\raggedright\arraybackslash}p{0.22\linewidth}
		>{\raggedright\arraybackslash}p{0.30\linewidth}
		>{\raggedright\arraybackslash}p{0.42\linewidth}
		@{}}
	\toprule
	\textbf{Use Case} & \textbf{Metrics Gathered} & \textbf{Impact Achieved} \\
	\midrule
	\endhead
	\bottomrule
	\endlastfoot
	
	AI-based Code Documentation Generation &
	Time spent on documentation creation
	
	Documentation quality
	
	Developer satisfaction with documentation process &
	90\% reduction in time spent on documentation creation
	
	22\% improvement in documentation quality
	
	55\% increase in developer satisfaction \\
	
	\midrule
	
	AI-driven Refactoring &
	Code complexity
	
	Code readability
	
	Developer satisfaction with refactoring process &
	28\% reduction in code complexity
	
	35\% improvement in code readability
	
	45\% increase in developer satisfaction \\
	
	\midrule
	
	Pull Request Review Automation &
	Automated pull request review achievement
	
	Time spent on pull request reviews
	
	Developer satisfaction with review process &
	Continuous automated operation confirmed via system logs
	
	75\% reduction in time spent on reviews
	
	55\% increase in developer satisfaction \\
	
	\midrule
	
	Internal LLM Chatbot for Knowledge Sharing &
	Time spent searching for information
	
	User satisfaction score
	
	Employee satisfaction regarding information access
	
	Onboarding satisfaction &
	40\% reduction in time spent searching for information
	
	User satisfaction score of 4/5 or higher achieved \\
	
\end{longtable}

The outlined use cases showcase the immense potential of AI adoption in software companies. The structured approach provided by the SAIL framework allows companies to tackle complex problems in a manageable way, breaking them down into smaller, iterative steps. Each use case demonstrates significant time and cost savings, as well as improvements in productivity and employee satisfaction.


Based on the gathered metrics, it is evident that AI adoption can lead to significant improvements. 
As a summary, in this chapter we can conclude that the outlined use cases are a perfect showcase of the immense potential of AI adoption in software companies. This chapter detailed the application of the SAIL framework to four distinct use cases, each demonstrating significant time and cost savings, as well as improvements in productivity and employee satisfaction. 
By incorporating management and engineering perspectives, the SAIL framework ensures that AI initiatives are strategically aligned, technically feasible, and effectively executed. The metrics gathered from each use case provide concrete evidence of the value that AI can bring to software companies when adopted in a structured manner. These findings set the stage for a deeper discussion on the implications of AI adoption in the subsequent chapter.


\chapter{Discussion}\label{chapter-7-discussion}

\section{Key Insights}\label{key-insights}

The entire objective of this thesis was to create a structured framework
which would help software companies to adopt AI in a structured manner
and increase the probability of success. The creation of the SAIL framework
serves as the first life-cycle framework which works on iterative steps
and adapts to each use case accordingly. The SAIL framework decomposes a
large AI initiative into 8 steps each with a quality gate being the
decision to proceed, iterate or stop. In regards to the insights from
the use cases, the most significant findings are the cleverness of the
framework to really granularize the complex problem of AI adoption into
smaller manageable pieces. The ease of the implementation of the
framework was one of the key success factors as it does not require a lot
of resources to implement it. Other frameworks which are focused on
project management such as PRINCE2 or Agile are very resource intensive
and require a lot of time and effort to implement them. The lightweight
nature of SAIL was one of the key success factors. The most resource-intensive stage of SAIL is the pilot execution phase, which in fact is
not actually resource-intense as the only resource-intense part is the
actual implementation of the pilot and thus it being the engineering
time, which would otherwise be spent on feature development. The
surprising success factor is the fact that the governance board does not
require a lot of time and effort as the decision making is very
straightforward and the quality gates are very clear.

The role of the governance board faded out when use cases utilize the same
or similar technology, as the risks and potential complications are
already known and therefore the governance board does not have to spend
a lot of time evaluating the risks. A great example for this is the usage
of GitHub Copilot for code generation and assistance. The risks and
potential complications are already known and therefore the governance
board does not have to spend a lot of time evaluating the risks.

In terms of insights which were gathered, it can be seen that the SAIL
framework is a great fit for problems which usually require a lot of
trial and error and are not straightforward. The more complex the
problem the more value SAIL is able to deliver as it breaks down the
complexity into smaller manageable pieces.


The outlined numbers showcase the immense potential of AI adoption in
software companies; this however is only possible to be achieved with a
structured approach which companies lack. If
multiple of these use cases are implemented at once and proper change
management is done, the impact on the productivity of the engineers and
the overall efficiency of the company is truly revolutionary. This marks
a historic moment for companies to leverage AI in order to create
sustainable competitive advantage.

A very important insight which was gathered is the fact that the SAIL
framework is not actually a guarantee of success. The best comparison
for SAIL is a GPS navigation system. It provides a structured approach
and a roadmap to the destination, however it does not guarantee that you
will reach the destination.

This type of mentality positions the SAIL framework as a means of
delivery and not as a final product. The success of the AI adoption
initiative still relies on the input of ideas and creativity of the
engineers and the management. That is one of the reasons why the SAIL
framework focuses on an ``Engineer-first'' mentality as engineers know
their daily struggles and pain points the best.

\section{Contributions to Theory and
	Practice}\label{contributions-to-theory-and-practice}


The SAIL framework is contributing to both theory and practice in several ways. From the perspective of theoretical contribution, the SAIL framework is linking to Dynamic Capabilities Theory, as it operationalizes the sensing, seizing and transforming capabilities in the process of AI adoption. The SAIL framework actually extends the current standard by converting both physical resources and digital assets into capabilities which are able to create value. The value in this case being a custom AI agent, which is able to solve a specific problem and thus create value for the company. 

From the perspective of innovation diffusion theory, the SAIL framework extends the current theory by providing concrete metrics and implementation steps rather than being too abstract.
The fact that the pilots under the SAIL framework emphasized on iteration, the probability for success is in fact higher due to a feedback loop gathered from each iteration. Each iteration and each pilot does not only serve as a proof of concept but also as a learning opportunity. The innovation diffusion theory works on the principle of trial and error and thus the SAIL framework embraces this principle by having multiple pilots rather than one big bang approach. 


The actual criticism of Innovation Diffusion Theory of being too abstract was addressed and proven wrong by providing concrete implementation steps and metrics which can be measured and tracked. As the SAIL framework emphasizes on practicality in terms of having clear steps and quality gates, it provides a clear roadmap for companies to follow when adopting AI. 

The great aspect of the SAIL framework is that it guides the development of new innovative branches rather than focusing on existing ones. The entire premise of it is to help CTOs and engineering leaders navigate the complex world of AI adoption and thus yield sustainable advantage from internal AI initiatives. By introducing quality gates, the SAIL framework introduces a level of transparency and a reality check which helps prioritize and rank high-impact AI projects and thus reduce the risk of implementing high effort low impact AI initiatives.

The structured approach allows engineering teams to start with brainstorming and putting an engineer-first mindset as a root cause of internal struggles. The structured approach also allows navigation during uncertainty via opportunity scouting and thus provides a replicable roadmap through multi-pilot design. 

It could be noticed that during the execution of the pilots the impact of the SAIL framework depends heavily on the creativity and quality of the use cases proposed. The quality of the output can be only as good as the input provided. As already mentioned, the SAIL framework does not guarantee success in terms of the impact on the organization; it only guarantees a structured approach towards AI adoption. The actual ROI is only guaranteed by the creativity and quality of the use cases proposed. 

The main impact on the daily lives of engineers is the fact that the engineers have a backbone and a structure to follow when proposing AI initiatives instead of having to figure it out on their own and work in a chaotic manner. 

The framework, however, does still have its limitations and areas of improvement which will be discussed in the next section.


\section{Limitations}\label{limitations}

This entire thesis has focused on introduction of the SAIL framework as a structured lifecycle framework for AI adoption in software companies. This implies that the framework is only applicable to software companies and not other industries. 
This might be seen as a limitation as other industries such as financial services, manufacturing and others might also be interested in AI adoption; however, due to the specifics of these industries the SAIL framework might not be a perfect fit due to increased regulatory requirements and slower pace of change. 

Another limitation is the rapid evolution of the AI landscape. AI landscape and technology is changing on daily basis and therefore it is very important to mention the fact that when a new use case starts to be developed with the SAIL framework it might happen that by the end of the development cycle the technology has already changed and therefore the initial assumptions are no longer valid. This is however not the limitation of the SAIL framework itself but rather a limitation of the AI world as a whole. The rapid evolution of new technology and design principles and paradigms is a challenge for any framework and suddenly the teams are entering unknown territory. 

What however is the current limitation of the SAIL framework is the fact that it's only composed out of eight steps. Someone might argue that it's too broad and generic and does not give concrete steps which need to be done. 
The SAIL framework is however designed to be still hands on and also apply to various different use cases. As the landscape of AI does not encompass only generative AI or machine learning algorithms or natural language processing, but also computer vision, autonomous agents and others, the SAIL framework had to be designed to still encompass all of these. 




\section{Implications and Next
	Steps}\label{implications-and-next-steps}

In terms of the implications and the next steps, we have to differentiate the implications for theory and practice. 
From the practical point of view, the SAIL framework is easy to apply and easy to understand. Based on the use cases analyzed throughout this entire thesis, it's ready to be deployed into organizations. The next steps, therefore, are going to be applying the SAIL framework in real-life organizations and gathering more data and feedback from the users.

From the theoretical point of view, it opens endless doors for further research and exploration. Extra research and exploration of each of the steps of the SAIL framework can be done in order to further improve the framework and make it more robust. Each of the steps can be further explored in order to provide more concrete steps and guidelines for companies to follow.

There is no limit in terms of how much further research can be done in order to improve the framework and make it more reliable and robust.

What would be considered as next steps would be research in terms of how to potentially expand and try to integrate the SAIL framework into already existing industries such as financial services, manufacturing and others. This would obviously require extra research, understanding of the industries, regulations, and specifics, but it would be a great next step in terms of implementing, revolutionizing and bringing innovation into more conservative and slower-paced industries.



What, however, has to be mentioned as a next step is the fact that the current project management frameworks such as PRINCE2 and Agile do not encompass AI adoption and therefore there is a great opportunity to integrate the SAIL framework into these already existing frameworks in order to provide a more holistic approach towards project management with AI adoption. This would be a great next step in terms of further research and exploration.





\section{Reflections on the framework and its
	application}\label{reflections-on-the-framework-and-its-application}

The SAIL framework represents a significant advancement in the
structured adoption of AI within software organizations. The major AI-based use cases which SAIL is perfect at employing and solving are use
cases which are focused on internal process optimization and enhancement
of productivity. 

In terms of all the use cases, the key pattern in them is either
code base optimisation or workflow optimisation. On the other hand, these use
cases are in fact the key processes which software companies are
focusing on in order to increase their productivity and efficiency.
Coding takes majority of the time of software engineers and therefore
any optimisation in this area is a direct increase in productivity. From
the managerial perspective the use cases for workflow optimisation and
improvement are mainly focused on orchestrating the engineering
workforce in the most efficient and sustainable manner. If managers are
able to achieve this better and faster with AI use cases the
productivity of the whole organisation increases.

Use cases which do not make sense to be implemented with the SAIL
framework are use cases which are not directly linked to internal
process optimisation or are too simple to be implemented with SAIL. A
perfect example for this is the usage of already existing AI-based tools
and simply enabling them. SAIL as a framework requires complexity and
therefore simple use cases such as turning on page summarisation in
Confluence for all engineers or enabling Grammarly for all employees do
not make sense to be implemented with SAIL. These 2 outlined use case
can be implemented by simply turning on the feature and therefore the
entire process of SAIL is not required as these processes do not need
granularization and decomposition into smaller manageable pieces.

SAIL is a perfect fit for medium to complex use cases which require a
structured progress for initiatives to be successful. Use cases which
might come with a level of risk and uncertainty are also a perfect fit
for SAIL as it has multiple quality gates, review stages and stakeholder
involvement. The complexity of the use case is directly proportional to
the value SAIL is able to deliver. The more complex the use case the
more value SAIL is able to deliver as it breaks down the complexity into
smaller manageable pieces and therefore increases the probability of
success.

\chapter{Conclusion and Final
	Thoughts}\label{chapter-8-conclusion-and-final-thoughts}

\section{Summary of Research}\label{summary-of-research}

This research is aimed at addressing the critical need for a structured
approach of AI adoption in software companies. The entire research is
trying to answer the question of ``How can software organizations
effectively adopt and scale AI technologies through a structured
lifecycle framework?'' and the simple answer to this question is the
SAIL framework. This framework was developed based on an 8-step
iterative process, each step having a clear quality gate with the main
objective of pushing the boundaries of the complex problem of AI
adoption into smaller manageable pieces. This type of a new,
revolutionary and innovative framework answers the research question as
it provides the structure and the roadmap for companies to follow in
order to successfully adopt AI technologies. The question is stated in a
way of how to adopt and scale AI technologies, and that is exactly what
SAIL encompasses. The core functionality of the framework being focus on
iterative pilots and rapid review cycles allows for quick learning and
adaptation, which is crucial in the fast-evolving AI landscape.

The key outcomes from the debate in chapter 7 clearly show the unique
value proposition of having a structured approach to AI adoption. The
outlined use cases demonstrate significant potential for an increase in
internal efficiency of processes and productivity of engineers. The
outlined metrics show everything which is possible to achieve with a
solid and structured approach. The concrete set of metrics such as time
savings, cost savings and other benefits provide a clear benchmark for
evaluating the success of AI initiatives.

The mentality of the framework being ``Engineer-first'' attribute of the
framework as its starting point is opportunity scouting by engineers.
The outcome of the framework will be only as good as its input. By
having high-quality input from engineers which understand their domain
the entire pipeline of SAIL will be benefited.

In order to give a clear summary of this research the SAIL framework can
be summarized as following:

\begin{itemize}
	\tightlist
	\item
	SAIL framework is a structured lifecycle framework for AI adoption in
	software companies.
	\item
	It consists of 8 iterative steps with clear quality gates working on
	the principle of ``Iterate, Proceed, Stop''.
	\item
	It guides companies from opportunity scouting to AI-native
	optimization by focusing on governance, strategic alignment, ability
	to pilot and scale initiatives.
	\item
	It is emphasizing an ``Engineer-first'' mentality which leverages the
	expertise of engineers
	\item
	The framework was successfully validated through 4 diverse use cases
	showcasing its versatility and applicability.
	\item
	SAIL provides a full replicable roadmap for medium to complex AI
	initiatives from scouting to scaling.
\end{itemize}

\section{Integrated Contributions of the SAIL
	Framework}\label{integrated-contributions-of-the-sail-framework}

The SAIL framework makes a very unique contribution in incorporating
multiple aspects of strategic, technical, and operational considerations
of AI adoption into a single cohesive navigable process. Unlike other
existing models, which are very much heavy and rigid, SAIL is
lightweight, flexible, iterative while still providing discipline
through its quality gates.

Its novelty lies in three core aspects:

\begin{enumerate}
	\def\labelenumi{\arabic{enumi}.}
	\tightlist
	\item
	Operationalization of AI Adoption as a Lifecycle Capability
\end{enumerate}

\begin{itemize}
	\tightlist
	\item
	SAIL treats AI adoption as a repeatable organizational capability
	rather than a one-off project, this meaning that it is creating a
	dynamic capability for the organization to continuously sense, seize
	and transform with AI technologies. This principle allows for new
	technology being continuously integrated into the organization as it
	evolves.
\end{itemize}

\begin{enumerate}
	\def\labelenumi{\arabic{enumi}.}
	\setcounter{enumi}{1}
	\tightlist
	\item
	Practical Roadmap with Built-in Metrics
\end{enumerate}

\begin{itemize}
	\tightlist
	\item
	By providing concrete steps and success metrics the iterative pilot
	strategy allows organizations to learn and adapt quickly while still
	observing and measuring impact of the solutions. This practical
	roadmap is a significant advancement over abstract models which lack
	actionable guidance and accountability.
\end{itemize}

\begin{enumerate}
	\def\labelenumi{\arabic{enumi}.}
	\setcounter{enumi}{2}
	\tightlist
	\item
	Bridging Strategy, Governance and Innovation
\end{enumerate}

\begin{itemize}
	\tightlist
	\item
	SAIL offers a unique balance between strategic alignment, governance
	oversight and innovation agility. It allows teams to experiment and
	pilot while ensuring alignment with broader organizational goals and
	risk management through governance involvement.
\end{itemize}

Together these three key aspects position SAIL as not only a theoretical
contribution but also a highly practical framework acting as an AI
enabler for software organizations.

\section{Limitations and Caveats}\label{limitations-and-caveats}

As with any great and innovative idea, the SAIL framework comes with its
own limitations and caveats.

\begin{itemize}
	\item
	The first limitation currently standing. In the scope of sale framework being used only for software companies. While inventing the sale framework The primary focus was software companies, and all the aspects of it were tailored precisely towards them. The limitation comes in the narrowness of the scope. And uncertainty whether? This framework will also fit. In other process industries.
	\item
	Second main limitation of the framework is its input-reliance. The
	framework and its impact on the company will be only as good as its
	input. If the engineers reate low-quality input the entire pipeline of
	SAIL will be affected in lower ROI and lesser impact on the company.
	The poor input might lead to poor output.
	\item
	Third limitation is the fact that SAIL is not a guarantee of success
	due to the aspect of change management and human factors. The
	framework provides a structured approach and a roadmap, however the
	guarantee to actually reach the destination is not there due to human
	nature of not trusting new technologies. This aspect is unfortunately
	out of the scope of SAIL and is a matter of change management and
	leadership. This dependency on human factors might come with a level
	of uncertainty and risk which is not accounted for in the framework,
	however might also extremely benefit if the change management is done
	properly and people have a natural curiosity and interest in new
	technologies.
\end{itemize}

\section{Practical Implications}\label{practical-implications}

For any organization whose scope is directly software And its chooses to apply the cell framework the implications and the benefits are truly revolutionary. 

By applying the sale framework, the organizations can expect to achieve:

\begin{itemize}
	\item
	Structured approach to AI adoption. Based on the outcome of this thesis, it can be stated that several teams have managed to adopt AI solutions In a structured manner by following sale framework. This structured approach minimized chaos optimized results and led to succesful adoption of AI technologies.
	\item
	Increase productivity and efficiency. Our use cases have demonstrated the potential of saving vast amounts of time and costs by automating significant processes, while not only freeing up engineers time but also increasing their satisfaction and engagement.
	
	\item
	Faster innovation cycles. The principle of introducing decision gates allows teams to iterate, reject, or proceed based on the current state of the use case. This gives the team the ability To be extremely. agile and adapt quickly to the fast-evolving AI landscape. The framework emphasizes on the principle of ``Fail-fast-learn-faster'' which is crucial in the current fast-moving AI world.
	
	\item
	Reduced uncertainty and risk. In decision gates inside of the framework Serve as a mirror in terms of reality check and progress evaluation. By having clearly established decision gates, the engineers and management are always up to date and are completely eliminating any personal biases and feelings towards a use case. As it often happens, managers try to commit to a Decision and stick to it through thick and thin, even if the reality is different. By having clear quality gates, the risk of committing to a poor use case is minimized.
	\item
	Increased innovation. The framework encourages engineers to take action, innovate and create new ideas which might be impactful for the business. By having structured approach. The company can leverage the ideas of engineers into a strategic position. Which elevates the entire company and pushes the boundaries of what is currently possible. 
\end{itemize}


The practical implications of SAIL extend beyond just the immediate
benefits of individual AI initiatives. The framework fosters a culture
of innovation, learning and curiosity within the organization. It tries
to position the company as a learning organization and emphasizes on the
importance of continuous improvement and mentality of ``Engineer-first''
and ``Fail-fast-learn-faster''. It puts down the fear of failure and
encourages experimentation and creativity as the key drivers of success
in the fast-evolving AI landscape.

It is important to note that the successful adoption of SAIL and
adoption of AI in the company truly lies in the commitment of the
leadership and the engagement of engineers. The framework on its own is
just a set of steps with quality gates, however the real success comes
from the people and their commitment to leverage this opportunity. The
framework is only a means of delivery and not the final product. It
might give all the opportunity and direction, however the actual journey
and destination is up to the people.

\backmatter
	
\bibliography{thesis_references}

\end{document}